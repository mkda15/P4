%\pdfbookmark[0]{English title page}{label:titlepage_en}
%\aautitlepage{%
%  \englishprojectinfo{
%    Project Title %title
%  }{%
%    Scientific Theme %theme
%  }{%
%    Fall Semester 2010 %project period
%  }{%
%    XXX % project group
%  }{%
%    %list of group members
%    Author 1\\ 
%    Author 2\\
%    Author 3
%  }{%
%    %list of supervisors
%    Supervisor 1\\
%    Supervisor 2
%  }{%
%    1 % number of printed copies
%  }{%
%    \today % date of completion
%  }%
%}{%department and address
%  \textbf{Electronics and IT}\\
%  Aalborg University\\
%  \href{http://www.aau.dk}{http://www.aau.dk}
%}{% the abstract
%  Here is the abstract
%}

\cleardoublepage
{\selectlanguage{english}
\pdfbookmark[0]{Titlepage}{label:titlepage_en}
\aautitlepage{%
  \englishprojectinfo{
    Time and frequency analysis of music signals
  }{%
	Signals and systems
  }{%
    Spring semester 2017%project period
  }{%
	Mattek4 G4-101a
  }{%
    Christian Hilligsøe Toft \\
    Frederik Appel Vardinghus-Nielsen \\
    Martin Kamp Dalgaard \\
    Trine Nyholm Jensen
  }{%
 	Peter Koch \\
 	Henrik Garde
  }{%
    6 % number of printed copies
  }{%
	26-05-2017
  }%
}{%department and address
  \textbf{Maths-Technology}\\
  Aalborg University \\
  \href{http://www.en.aau.dk/}{http://www.en.aau.dk/}
  % \href{http://www.aau.dk}{http://www.aau.dk}
}{% the abstract
This report treats the problem of automatically transcribing a piece of music to a note sheet and seeks to develop an algorithm for this purpose. In order to do so the report uses the \textbf{Fourier transform} as a mathematical tool to perform \textbf{time and frequency analysis} of music signals recorded in an \textbf{anechoic room} at Aalborg University. The theory of \textbf{discrete-time systems} is described, and a \textbf{filter with finite impulse response} is designed for the purpose of filtering out \textbf{noise} added to the recorded signals. Furthermore, the trade-offs, advantages, and disadvantages of the Fourier transform and the implemented filter are considered. The frequencies appearing in the signals over time are visualised with a \textbf{spectrogram} generated with the use of a \textbf{short-time Fourier transform} and, finally, an \textbf{algorithm} capable of recognising significant frequencies in the music signals, so as to create a note sheet, is implemented. This algorithm is shown to be able to detect the frequencies of single tones within $\pm 5$ Hz.}}