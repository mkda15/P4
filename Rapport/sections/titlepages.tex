%\pdfbookmark[0]{English title page}{label:titlepage_en}
%\aautitlepage{%
%  \englishprojectinfo{
%    Project Title %title
%  }{%
%    Scientific Theme %theme
%  }{%
%    Fall Semester 2010 %project period
%  }{%
%    XXX % project group
%  }{%
%    %list of group members
%    Author 1\\ 
%    Author 2\\
%    Author 3
%  }{%
%    %list of supervisors
%    Supervisor 1\\
%    Supervisor 2
%  }{%
%    1 % number of printed copies
%  }{%
%    \today % date of completion
%  }%
%}{%department and address
%  \textbf{Electronics and IT}\\
%  Aalborg University\\
%  \href{http://www.aau.dk}{http://www.aau.dk}
%}{% the abstract
%  Here is the abstract
%}

\cleardoublepage
{\selectlanguage{english}
\pdfbookmark[0]{Titlepage}{label:titlepage_en}
\aautitlepage{%
  \englishprojectinfo{
    Time and frequency analysis of music signals
  }{%
	Signals and systems
  }{%
    Spring semester 2017%project period
  }{%
	Mattek4 G4-101a
  }{%
    Christian Hilligsøe Toft \\
    Frederik Appel Vardinghus-Nielsen \\
    Martin Kamp Dalgaard \\
    Trine Nyholm Jensen
  }{%
 	Peter Koch \\
 	Henrik Garde
  }{%
    4 % number of printed copies
  }{%
	26-05-2017
  }%
}{%department and address
  \textbf{Maths-Technology}\\
  Aalborg University \\
  \href{http://www.en.aau.dk/}{http://www.en.aau.dk/}
  % \href{http://www.aau.dk}{http://www.aau.dk}
}{% the abstract
This report considers \textbf{time and frequency analysis} of music signals through the \textbf{Fourier transform}. \textbf{Discrete-time systems} are described, and a \textbf{filter with finite impulse response} is designed for the purpose of filtering out \textbf{noise}. Furthermore, the trade-offs, advantages, and disadvantages of the Fourier transform and the implemented filter are considered. Finally, an \textbf{algorithm} capable of recognising the \textbf{frequencies} in the music signals is implemented.}}