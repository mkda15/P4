\section{Definition of wavelets}
Wavelet theory provides a way of constructing orthonormal bases in $\mathcal{L}^2(\mathbb{R})$, which is a concrete Hilbert space consisting of functions. These functions are scaled and translated versions of a fixed function. For an orthonormal basis $\{\textbf{e}_k\}$ of $\mathcal{L}^2(\mathbb{R})$ all functions $f \in \mathcal{L}^2(\mathbb{R})$ have an expansion
\begin{align*}
f = \sum_{k\in\mathbb{Z}}^\infty c_k \textbf{e}_k
\end{align*}

for suitable coefficients $c_k$. The index set of the sum is generally infinite but in order for the representation to be of practical use, the functions $f$ must be able to be approximated well by finite partial sums \cite{page 160, FSE2010}.

\begin{definition}[Wavelet]
Let $\psi \in \mathcal{L}^2(\mathbb{R})$.
\begin{enumerate}
\item For $j,k \in \mathbb{Z}$, define the function $\psi_{j,k}$ by
\begin{align*}
\psi_{j,k}(t) = 2^{j/2} \psi(2^jt-k), \quad t \in \mathbb{R}.
\end{align*}
\item The function $\psi$ is called a wavelet if the functions $\{\psi_{j,k}\}_{j,k\in\mathbb{Z}}$ form an orthonormal basis for $\mathcal{L}^2(\mathbb{R})$.
\end{enumerate}
\end{definition}

Therefore, $\psi_{j,k}$ is a dilation and translation as defined in definition \ref{def:TMD} and may be written in terms of the translation operator $T_k$ and the dilation operator $D$ as \cite{page 160, FSE2010}
\begin{align*}
\psi_{j,k} = D^j T_k \psi, \quad j,k \in \mathbb{Z}.
\end{align*}

A simple wavelet is the Haar wavelet.

\begin{definition}[Haar Wavelet] \label{HaarWave}
The \textit{Haar function} is defined by
\begin{align*}
\psi(t) =
\begin{cases}
1 \quad &\textnormal{if } 0 \leq t < \frac{1}{2} \\
-1 \quad &\textnormal{if } \frac{1}{2} \leq t < 1 \\
0 \quad &\textnormal{otherwise}
\end{cases}
\end{align*}
\end{definition}

The proof that the functions $\{\psi_{j,k}\}_{j,k\in\mathbb{Z}}$ constitute an orthonormal basis for $\mathcal{L}^2$ is quite technical and will therefore be skipped as the purpose of this chapter is merely to introduce and apply wavelet analysis as an alternative to the Fourier transform \cite{page 161, FSE2010} \martin{Overvej dette. \textregistered}.
\\ \\
In general, $\psi(t)$ is compactly supported, which means that $\psi(t) \neq 0$ only inside a bounded range $a < t < b$. Furthermore, $\psi(t)$ has zero mean, which means that it takes both positive and negative values:
\begin{align*}
\int_{-\infty}^\infty \psi(t) dt = 0.
\end{align*}

Therefore, since a wavelet is nonzero only within a finite range and the mean is zero, then $\psi(t)$ has the form of a wave, which is the reason why $\psi(t)$ is called a wavelet \cite{page 411, Wang}. The Haar wavelet is discontinuous at $t = 0, \frac{1}{2}, 1$. Examples of other wavelets that are continuous are the Shannon, Mortlet, and Marr wavelets \cite{page 417-420, Wang}.
\\ \\
The following definition is the background for a main theorem of this section \cite{page 170, FSE2010}.

\begin{definition}[Vanishing moments]
Let $N \in \mathbb{N}$. A function $\psi$ has $N$ vanishing moments if
\begin{align*}
\int_{-\infty}^\infty x^\ell \psi(x) dx = 0 \quad for \quad \ell = 0, 1, \dots, N-1.
\end{align*}
\end{definition}

The Haar wavelet defined in definition \ref{HaarWave} has only one vanishing moment. If $\psi$ has a large number of vanishing moments, the following result shows that only relatively few coefficients $\langle f, \psi_{j,k} \rangle$ will be large and therefore only a small number of wavelets are needed in the expansion.

\begin{theorem}[Decay of wavelet coefficients]
Assume that the function $\psi \in \mathcal{L}^2(\mathbb{R})$ is compactly supported and has $N$ vanishing moments. Then, for any $N$ times differentiable function $f \in \mathcal{L}^2(\mathbb{R})$ for which the $N$th derivative $f^{(N)}$ is bounded, there exists a constant $C > 0$ such that
\begin{align} \label{eq:decay_wave_coeff}
|\langle f, \psi_{j,k} \rangle| \leq C 2^{-jN} 2^{-j/2}, \quad \forall \ j \geq 1, \quad k \in \mathbb{Z}.
\end{align}
\end{theorem}

\eqref{eq:decay_wave_coeff} suggests that a high number of vanishing moments $N$ implies that the numbers $\langle f, \psi_{j,k} \rangle$ decay quickly as $j \to \infty$. Therefore, the higher the number of vanishing moments a wavelet has, the fewer coefficients $\{ \langle f, \psi_{j,k} \rangle \}_{j\in\mathbb{N},k\in\mathbb{Z}}$ are needed in the expansion \cite{page 170, FSE2010}.
%\\ \\
%Mere teori til dette kapitel (ikke nødvendigvis det hele):
%\begin{enumerate}
%\item Continuous-Time Wavelet Transform (CTWT).
%\item Discrete-Time Wavelet Transform (DTWT).
%\item Filtrering ved hjælp af wavelets.
%\item ``Multiresolution Analysis and Discrete Wavelet Transform'' (kapitel 10 i R. Wang).
%\end{enumerate}