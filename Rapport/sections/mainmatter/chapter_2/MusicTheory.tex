In this chapter basic music theory necessary for the understanding of the report is presented.
\\ \\
An example of a staff system with 15 notes is shown in figure \ref{fig:cmajor}. A note in such a system is generally a symbolic representation of a certain pitch, which is associated to a particular frequency (specified in hertz, Hz). Each note thus contains information about the frequency of the associated pitch and also about its duration in time. A note sheet is therefore actually a symbolic representation of a diagram over time and frequency, which is also called a spectrogram.

\begin{figure}[H]
    \centering
    \includegraphics[width = 0.6\textwidth]{figures/Cmajor.png}
    \caption{Example of notes in a staff system.}
    \label{fig:cmajor}
\end{figure}

\noindent
A staff system consists of 5 horizontal lines whereupon the notes are placed. Basically, only 12 key notes are repeated throughout the system but each with different frequencies. The key notes with the same names differ by an interval of $n$ octaves for some integer $n$ and are said to be octave equivalent. \cite{MusicTheory} The relation in frequency between two pitches $A_0$ and $A_n$ are thus $A_n = 2^n A_0, n \in \mathbb{Z}$, and $A_n$ is therefore $n$ octaves above $A_0$.
\\ \\
The notes in the staff system are named by the letters $A$-$G$ depending on their vertical position. After $G$ the naming is repeated with the next $A$, which lies an octave above the previous one. Furthermore, between several of these pitches lie semitones, which are noted as e.g. $A$\hashsharp{}. Table \ref{tab:freq} shows all pitches in a single octave.
\\ \\
To be able to distinguish between octave equivalent pitches they are e.g. noted as $A_4$ and $A_5$. $A_4$ is also called the concert pitch and is normally chosen to be $440$ Hz by definition but may vary by $\pm 3$ Hz depending on who's playing. All other pitches can be defined from the chosen frequency of $A_4$ since it is the $12^{th}$ key note after $A_3$, whose frequency is half the size of $A_4$'s. The relation in frequency between two neighbouring pitches, e.g. $A$ and $A$\hashsharp, is therefore $A$\hashsharp \ $= \sqrt[\leftroot{-2}\uproot{2}12]{2} \cdot A$. \cite{MusicTheory} The frequencies of the pitches in the interval $A_3$-$A_4$ rounded off to the nearest integer is shown in table \ref{tab:freq}.
\\ \\
When a pitch is played the pitch's overtones will sound along with the actual pitch. The frequency of a pitch's overtone is an integer multiple of the original pitch's frequency, and this phenomena is therefore among others reflected through the octave equivalent pitches of the original one. However, the pitch $A_2$ with frequency $110$ Hz also has the overtone $E_3$ ($330$ Hz) along with the octave equivalent pitches $A_3$ ($220$ Hz), $A_4$ ($440$ Hz) and $A_5$ ($880$ Hz). If several instruments play at the same time all the original pitches and overtones are mixed, which is one of several reasons why it is difficult to translate music played by several instruments into a single note sheet. On the other hand, if only one instrument is playing, the overtones are not mixed together, and the pitch is thus easily distinguished from the overtones. The pitch in question will then usually be the one with the lowest frequency. The overtones therefore complicate the determination of the original pitch, but the determination is not impossible if the pitch e.g. is played alone in an anechoic room. This is shown in figure (insert reference to figure below).

\begin{table}[H]
\centering
\caption{Frequencies of the pitches $A_3$-$A_4$ shown in Hz and rounded off to nearest integer.}
\label{tab:freq}
\begin{tabular}{|l|c|c|c|c|c|c|c|c|c|c|c|c|c|}
\hline
Pitch & $A_3$ & $A_3$\hashsharp & $B_3$ & $C_3$ & $C_3$\hashsharp & $D_3$ & $D_3$\hashsharp & $E_3$ & $F_3$ & $F_3$\hashsharp & $G_3$ & $G_3$\hashsharp & $A_4$ \\ \hline
Freq & 220 & 233 & 247 & 262 & 277 & 294 & 311 & 330 & 349 & 370 & 392 & 415 & 440 \\ \hline
\end{tabular}
\end{table}

\noindent
*Insert spectogram showing a single pitch with overtones here.*
\\ \\
Playing a pitch alone in an anechoic room is a way of minimizing noise factors. There are many different noise sources from e.g. the surrounding environment, electrical equipment, other people and even the instrument playing itself since it is difficult to make a pure tone by an acoustic sound source such as a guitar. A pure tone is a sinusoidal waveform consisting of a single frequency and may therefore be difficult to play on an instrument. \cite{AcousticNoise} Usually, sound is reflected off the walls in a room which is also a source of noise. This is a form of folding and is minimized in the anechoic room because sound is absorbed by the walls. Moreover, due to the construction of the anechoic room, noise from e.g. bypassing cars is also minimized, and the sound may be recorded with a minimum of hardware which otherwise may also produce noise.
\\ \\
The music used in this project will therefore be recorded in an anechoic room because the noise is minimized. It is not expected that musicians using the system in the future have access to an anechoic room as well but if the system doesn't work with sounds recorded in the anechoic room then it with most certainty doesn't work at other places neither. However, in order to reproduce the conditions of a typical musician working around other people, background noise can also be made in the anechoic room but should of course not drown the music. This form of noise is additive whereas e.g. noise reflected off walls as mentioned above is multiplicative. In general, multiplicative noise depends on the state of the system whereas additive noise does not. Therefore, the output $x[n]$ of the sampled data $s[n]$ corrupted by the additive noise $a[n]$ is $x[n] = s[n] + a[n]$ whereas the equation for the multiplicative noise $m[n]$ is $x[n] = s[n]m[n]$.
\\ \\
Comment on this chapter: does it need a conclusion? Or is there something else missing? Which one of the words "pitch" and "tone" is better to use? According to Wikipedia, "tone" is the best word (\url{https://en.wikipedia.org/wiki/Musical_tone}) but I personally  prefer "pitch" over "tone". \textregistered 

%Dansk tekst
%I dette kapitel introduceres grundlæggende musikteori, der er nødvendig for forståelsen af projektet.
%\\ \\
%I nodesystemet er en node en symbolsk repræsentation af en bestemt tone, som er tilknyttet en specifik frekvens (angivet i hertz, Hz). Hver node indeholder således information om tonens højde og desuden også om dens varighed. Et nodeark er således en symbolsk repræsentation af et diagram over tid og frekvens, hvilket også kaldes et spektrogram.
%\\ \\
%Til at notere noder benytter man et nodesystem, der består af fem vandrette linjer, hvor noderne er placeret. Grundlæggende findes der 12 forskellige toner, som gentages igennem nodesystemet. Principielt set findes der således flere toner med samme navn men med forskellige frekvenser, og disse toner siges at være oktavækvivalente, da den ene ligger et antal oktaver over den anden. Forholdet i frekvens mellem 2 oktavækvivalente toner $A_0$ og $A_n$ er derfor $A_n = 2^n A_0, n \in \mathbb{Z}$, og $A_n$ er således $n$ oktaver over $A_0$.
%\\ \\
%Noderne i nodesystemet består først og fremmest af stamtoner, som navngives med bogstaverne $A$-$G$ afhængigt af deres vertikale placering. Efter $G$ starter navngivningen forfra med det næste $A$, der ligger en oktav over det forrige. Mellem flere af stamtonerne ligger desuden andre toner, som sammen med stamtonerne udgør de 12 grundtoner. Afstanden mellem alle grundtonerne er en halv tone, og tonen mellem $C$ og $D$ noteres som C\hashsharp{}.
%\\ \\
%For at skelne mellem de oktavækvivalente toner noteres de for eksempel som $A_4$ og $A_5$. $A_4$ kaldes også kammertonen og vælges per definition normalt til at være $440$ Hz, men kan dog afvige med $\pm 3$ Hz afhængigt af den enkelte musiker eller orkester, der spiller musikken. Alle andre toner kan defineres ud fra den valgte frekvens for $A_4$, da $A_4$ er den 12. grundtone efter $A_3$, som har halvt så stor frekvens som $A_4$. Forholdet i frekvens mellem to naboliggende toner, for eksempel $C$ og $C$\hashsharp, er således $C$\hashsharp \ $= \sqrt[\leftroot{-2}\uproot{2}12]{2} \cdot C$. \cite{MusicTheory} Ud fra dette forhold ses det, at tonesystemet bygger på en logaritmisk skala. Frekvenserne for tonerne $A_3$-$A_4$ afrundet til nærmeste heltal er vist i tabel \ref{frektoner}.
%\\ \\
%Når en tone spilles vil tonens overtoner klinge sammen med den pågældende tone. En tones overtone er et heltals-multiplum af den oprindelige tone, og dette fænomen kommer således blandt andet til udtryk for de oktavækvivalente toner af den oprindelige tone. Hvis flere instrumenter spiller sammen blandes overtonerne sammen, hvilket er en vigtig årsag til, at det er problematisk at oversætte optaget musik spillet af flere instrumenter til noder. Hvis kun ét instrument spiller blandes overtonerne derimod ikke sammen, og så vil den spillede tone som udgangspunkt være tonen med den laveste frekvens. Overtonerne komplicerer derfor bestemmelsen af tonen, men det er ikke umuligt, hvis tonen spilles alene i for eksempel et lyddødt rum. Dette fænomen ses på spektogrammet i figur (indsæt reference).
%\\ \\ \\ \\
%\begin{table}[H]
%\centering
%\caption{Frekvensen for tonerne $A_3$-$A_4$ afrundet til nærmeste heltal.}
%\label{frektoner}
%\begin{tabular}{|l|c|c|c|c|c|c|c|c|c|c|c|c|c|}
%\hline
%Tone & $A_3$ & $A_3$\hashsharp & $B_3$ & $C_3$ & $C_3$\hashsharp & $D_3$ & $D_3$\hashsharp & $E_3$ & $F_3$\hashsharp & $F_3$ & $G_3$ & $G_3$\hashsharp & $A_4$ \\ \hline
%Frekvens & 220 & 233 & 247 & 262 & 277 & 294 & 311 & 330 & 349 & 370 & 392 & 415 & 440 \\ \hline
%\end{tabular}
%\end{table}
%
%
%
%
%
%
%*Indsæt spektogram af en enkelt spillet tone med tilhørende overtone her.*