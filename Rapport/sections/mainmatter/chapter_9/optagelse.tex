\section{Recording of audio file}
(maybe for appendix)\\
For both unit tests and test on the final system an known audio file are needed as specified. This section accounts for the process of recording the audio in the acoustic lab.
\subsection{Description}
Three different audio files are needed as a result of the recording.
\begin{itemize}
\item[1.] single note  
\item[2.] 10 seconds of single notes  
\item[3.] second file plus certain noise
\end{itemize} 
It is important to know the exact signal on each file, in order to use the files as reference points for the validation tests of the system. Due to this the recording of the audio will take place in the anechoic room provided by the acoustic laboratory at AAU. \\
\subsubsection{Anechoic room}   
An anechoic room is designed to absorb all reflections from sound or electromagnetic waves. Further the room is isolated from exterior noise. Which makes it possible to only record or measure the exact sound that is coming from the instrument, without the presence of interfering reflections. \\ The specific room at AAU is build as a box inside a box. The inner box is placed on rubber suspensions and the inner walls are covered by sound absorbing wedges of about 0.4m length, due to requirements for anechoic performances. The inside dimensions of the room are 4.5m times 5.0m with a height of 4.0m.

\subsection{Procedure}
The arrangement of the recording are sketched on figure \ref{}. The recording are performed using ... which fulfil the specifications from chapter \ref{ch3} to ensure that the signal are reproduced exact.\\ 
... insert drawing ...

\subsubsection{Equipment list}

\subsection{Source of error}
\begin{itemize}
\item[-] internal noise, from mic 
\item[-]	
\end{itemize}

 

