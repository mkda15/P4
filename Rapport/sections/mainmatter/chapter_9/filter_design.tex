\section{Summary} \label{sec:filtervalg}
In this section, the frequency analysis of the music and noise described in the former sections is used to form the basis for the design of the filter, of which the aim is to filter out the noise. The noise is both low- and high-frequency, whereas the frequency of the music lies between 82 Hz and 1000 Hz, which means that the filter should be a bandpass filter. The passband of the filter is determined from the frequency analysis of the music signal. It is important to preserve the frequencies that contain the most energy but it is not necessary to include all the harmonics that appears in the spectrum. By this, a passband between frequencies 70 Hz and 1000 Hz would be sufficient. While the guitar can play at frequencies up to 1047 Hz such high frequencies are not played in the recordings.\\\\
From the discussion of different filters and designs of these in chapter \ref{ch8} the implemented filter is chosen to be a FIR filter due to its easy implementation method. The trade-off of a FIR filter is its computational complexity, which is problematic if the filter is supposed to run in real-time -- that is, while the music is being played. However, in this project the music is first recorded and then analysed, which means that the number of computations is not a big factor. For an actual system working in real-time the number of computations is a bigger factor, and such a system could be considered to use an IIR filter.\\
As described in section \ref{subsec:FIR} the FIR filter will be computed by the window method, and the Kaiser window is chosen due to its capability to adapt according to the filter specifications.     