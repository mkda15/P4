\section{Filter design}
In this section, the frequency analysis of the music and noise in the former sections form the basis for the design of the filter, whose aim is to filter out the noise. The noise is low-frequency compared to the frequency of the music, which means that the filter should be a highpass filter \martin{Er der også nogle højfrekvente signaler, som vi gerne vil filtrere væk? Ellers er det jo ikke nødvendigt med et båndpas-filter. \textregistered}. From the discussion of different filters and designs of these in chapter \ref{ch7} the implemented filter is furthermore chosen to be a FIR filter since this type of filter is guaranteed to have linear phase, which preserves the wave form of the original signal, which of course is important in this project. Furthermore, FIR filters are rather easy to design. On the downside, a FIR filter uses many computations, which is problematic if the filter is supposed to run in real time - that is, while the music is being played. However, the music has already been recorded in this project, which means that the computations are not a big factor. An actual system working in real time could be considered to use an IIR filter due to the otherwise possibly high amount of computations from the FIR filter but the linear phase from the FIR filter is weighted as more important in this project.