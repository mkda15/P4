\section{Summary} \label{sec:filtervalg}
In this section, the frequency analysis of the music and noise in the former sections form the basis for the design of the filter, whose aim is to filter out the noise. The noise is both low- and high-frequency, whereas the frequency of the music lies in the middle, which means that the filter should be a bandpass filter. From the discussion of different filters and designs of these in chapter \ref{ch8} the implemented filter is furthermore chosen to be a FIR filter for its easy implementation method. The tradeoff of a FIR filter is its computational complexity, which is problematic if the filter is supposed to run in real time - that is, while the music is being played. However, in this project the music is first recorded and then analysed, which means that the computations are not a big factor. For an actual system working in real time the amount of computations plays a bigger factor, and such a system could be considered to use an IIR filter.