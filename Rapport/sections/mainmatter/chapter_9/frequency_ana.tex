\section{Frequency analysis of music}
In this section the recordings of music will undergo frequency analysis, and the goal is to be able to express which frequencies the tones played in the recordings are generally found at.

\subsection{Single tones}
Firstly, the tuning of guitar is checked for consistency. The low and high E strings on the guitar should vibrate and emit sounds frequencies of 82.41 Hz and 329.63 Hz, respectively. Figures (* insert references *) show the frequency spectra for the recordings of the two tones.
\begin{center}
* Insert plots of frequency spectra *
\end{center}

The most significant frequencies in the two recordings are 163.82 Hz and 329.83 for the low and high E's, respectively. As $163.82$ Hz $\approx 2\cdot82.41$ Hz this is regarded as an harmonic of the low E. The harmonics are moreover observable in the figures as reduced peaks at integer multiples of the fundamental frequencies of the tones. It is furthermore seen from the figures, that the energy in the signals is mainly located at frequencies above 75 Hz and below 1000 Hz for the low E and above 100 Hz and below 2000 Hz for the high E.

\subsection{Chords}
Figure (* insert reference *) and figure (* insert reference *) show the frequency spectra of the recordings of low and high E chords, respectively.
\begin{center}
* Insert figures *
\end{center}

The most significant frequencies are 163.86 Hz and 119.27 Hz. Once again it is assumed that the higher frequency of the low pitch E is due to harmonics. These frequencies do furthermore not correspond to a specific note - this is assumed to be because of the composition of chords being of multiple tones. The majority of the energy in the signals is located above 80 Hz and below 2000 Hz.

\subsection{Scale}
Figure (* insert references *) shows the frequency spectrum of playing a octatonic scale.
\begin{center}
* Insert plots *
\end{center}

The majority of the energy in the signal is as seen located above 100 Hz and below 600 Hz.

\subsection{Melody with single notes}
Figures (* insert reference *) show the frequency spectrum for a melody consisting only of single tones played slowly.
\begin{center}
* Insert plots *
\end{center}

The majority of energy in the signal is located above 100 Hz and below 2000 Hz.

\subsection{Melody with chords}
In figures (* insert reference *) is seen the frequency spectrum for a melody consisting only of single tones played slowly.
\begin{center}
* Insert plots *
\end{center}

The majority of energy in the signal is located above 90 Hz and below 800 Hz.

\section{Frequency analysis of noise}