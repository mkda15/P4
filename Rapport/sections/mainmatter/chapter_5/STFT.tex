\section{The Short-Time Fourier Transform}
A signal in time contains informations about the amplitude of the signal at specific times but does not hold any explicit informations about the frequencies. On the other hand, the Fourier transform of a signal contains information about a certain frequency but not at which time it appears.
\\ \\
From a musical point of view, let $f(t)$ be a piece of music describing the amplitude of the vibration of a speaker membrane in time, $t \in \mathbb{R}$. From this signal one is possibly able to detect the rythmical patterns of the music but probably not the melody or just single tones. On the other hand, the Fourier transform of $f$, $\mathcal{F}\{f(t)\}(\omega)$, will show the dominating frequencies of the piece of music from which one can detect the tones but not the duration of these. Therefore, neither the signal in time $f(t)$ nor the Fourier transform $F(\omega)$ in frequency contains all the relevant information in order to describe the signal.
\\
Figure \ref{fig:sine_STFT} is a follow-up from figure \ref{fig:sine_sum} and initially shows a sine wave of the form $f_1(t) = \sin(220\pi\cdot t) + \sin(440\pi\cdot t)$, which is the tone $A_2$ with frequency $110$ Hz and a single harmonic of $A_3$ with frequency $220$ Hz. After approximately 22 milliseconds the sine wave shifts to $f_2(t) = \sin(880\pi\cdot t) + \sin(1760\pi\cdot t)$, which is the tone $A_4$ with frequency $440$ Hz and a single harmonic of $A_5$ with frequency $880$ Hz.

\begin{figure}[H]
    \centering
    \includegraphics[width = 0.6\textwidth]{figures/ sine_STFT.png}
    \caption{Shifting of a tone.}
    \label{fig:sine_STFT}
\end{figure}

\martin{Indsæt eventuelt figurer magen til dem på side 23 i Gröchenig. \textregistered}
\\
Amazingly, the human ear and brain is able to perceive the bare signal $f$ and process it into a representation that provides simultaneous information about both time and frequency. This is actually what human beings call music and is represented through the score shown in figure \ref{fig:Cmajor} in chapter \ref{ch2}. The goal of time-frequency analysis (at least in this project) is to imitate the ear and create a joint time-frequency representation of a signal \cite{page 22, FTFA}. This is the incentive for the short-time Fourier transform (STFT), which may be thought of as the mathematical analogue of the musical score \cite{page 37, FTFA}. The following is inspired by \cite{page 37, FTFA}.
\\ \\
The idea in the STFT is to obtain properties of a local frequency spectrum of $f$ by restricting it to an interval and taking the Fourier transform of this interval. $f$ is restricted on this interval by using a window function, which is close to 1 near the origin and decays towards zero at the edges, which is therefore a smooth cut-off function. The boundaries created by a sharp cut-off function will be interpreted by the Fourier transform as a discontinuity or an abrupt variation of the signal, which is obviously not desirable \cite{Davis}. The following definition defines the STFT.
\\ \\
\begin{definition}{The short-time Fourier transform}
Let $f,g \in \mathcal{L}^2(\mathbb{R}^d)$. For a fixed window function $g \neq 0$ the short-time Fourier transform of a function $f$ with respect to $g$ is defined as:
\begin{align}
V_gf(\tau,\omega) = \int_{\mathbb{R}^d} f(t) \overline{g(t - \tau)} e^{-2\pi j t \omega} dt \quad \text{ for } \tau,\omega in \mathbb{R}^d
\end{align}
\end{definition}

- The discrete STFT
- The Uncertainty Principle: page 26 and proof?