In numerical practice, the equicalent of a given function f(x) of the continuous independent variable x is a finite number N of data values associated with a number t, customarily taken as running from 0 to N-1.

The discrete Fourier transform F(v) is defined by:
\begin{align*}
	F(v)= N^{-1}\sum_{\tau = 0}^{N-1}f(\tau)e^{-j \omega \frac{v}{N}\tau}
\end{align*}

Where the quantity v, is an integer going from %$0 \to N-1$

The discrete Fourier transform (DFT), is the discrete time variant of the Fourier transform, can be used to transform sample into from the time domain, into the frequency domain.
\\\\
Multiple parts of the DFT are analog to the continues version.
\\\\
Let a function in time $f(t)$ be a continuous signal.
If the function is sampled N times it could be denoted
$f[0],f[1],\dots,f[k],\dots,f[N-1]$.
\\\\
The Fourier Transform of $f(t)$ would be as in \ref{sec:cunt_Fourier_transform}.
\begin{align*}
	F(\omega) = \int_{-\infty}^\infty f(t)e^{-j\omega t} dt
\end{align*}

If you regard each sample $f[k]$ as a single impulse, with a giving area $f[k]$. Then the integrand only exist at the sample points:

\begin{align*}
	F(\omega) 
	&= \int_0^(N-1)T f(t) e^{-j\omega t}dt\\
	&= f[0]e^{-j0} + f[1]e^{-j\omega T} + \dots + f(N-1) e^{-j\omega(N-1)T}	
\end{align*}

or alternatively
\begin{align*}
	F(\omega) = \sum_{k=0}^{N-1}f[k]e^{-j\omega k T}
\end{align*} 

\chr{Går ud fra at der i Fourier transformationen står at den kan evalueres i et endeligt interval i stedet fra uendeligt til uendeligt.}

The DFT have to take into consideration that it's only feed a finite number of input data. 
In the style the continues case of the Fourier transform, where it's possible to evaluate it over a finite interval.
