This chapter describes the mathematical tools necessary to describe and understand the theory described in the following chapters. This project deals with functions and sequences in the so-called $\mathcal{L}^p$ and $\ell^p$ spaces, respectively. The following definitions are inspired by \cite{chapter 5-6, FSE2010} and \cite{chapter 2, FSP} and only considers a subset $D \subset \mathbb{R}$ due to the focus of this project. However, similar definitions also hold for all of $\mathbb{R}$.

\begin{definition}[The $\mathcal{L}^p$ spaces] \label{def:Lpspace}
Let $D \subset \mathbb{R}$ be a subset. $\mathcal{L}^p(D)$ is the set of absolutely Lebesgue-integrable functions over $D$ for any $p \in [1,\infty)$:
\begin{align*}
\mathcal{L}^p(D) = \left\{ f: D \to \mathbb{C}: \int_D |f(t)|^p dt < \infty \right\}.
\end{align*}

The normed vector space $\mathcal{L}^p(D)$ is the subspace consisting of vectors with finite $\mathcal{L}^p$ norm:
\begin{align*}
\|f\|_p = \left( \int_D |f(t)|^p dt \right)^{1/p}.
\end{align*}

Collectively, the $\mathcal{L}^p$ spaces are all Banach spaces.
\\ \\
It is furthermore required that the function $f$ is measureable. This will not be elaborated further in this project as it is beyond its focus.
\end{definition}

A specific case of interest in this project is for $p = 2$.

\begin{definition}[The $\mathcal{L}^2$ space] \label{def:L2space}
Let $D \subset \mathbb{R}$ be a subset. $\mathcal{L}^2(D)$ is the set of all functions on $D$ whose squares are absolutely Lebesgue-integrable over $D$:
\begin{align*}
\mathcal{L}^2(D) = \left\{ f: D \to \mathbb{C}: \int_D |f(t)|^2 dt < \infty \right\}.
\end{align*}

$\mathcal{L}^2(D)$ is the normed vector space of square-integrable complex-valued functions. The inner product is:
\begin{align*}
\langle f,g \rangle =  \int_D f(t) \overline{g(t)} dt,
\end{align*}

where $\overline{g(t)}$ is the complex conjugate of $g(t)$. The norm is:
\begin{align*}
\|f\|_2 = \left( \int_D |f(t)|^2 dt \right)^{1/2}.
\end{align*}

The $\mathcal{L}^2$ space is a complete and normed space with an inner product and is therefore a Hilbert space.

\end{definition}

The following definition defines linear operators on $\mathcal{L}^2(\mathbb{R})$ \cite{page 120, FSE2010}.

\begin{definition}[Translation, modulation, dilation] \label{def:TMD}
Consider the following classes of linear operators on $\mathcal{L}^2(\mathbb{R})$:
\begin{enumerate}
\item For $a \in \mathbb{R}$, the operator $T_a$, called translation by $a$, is defined by
\begin{align*}
T_a f(t) = f(t-a), \quad t \in \mathbb{R}.
\end{align*}

\item For $b \in \mathbb{R}$, the operator $E_b$, called modulation by $b$, is defined by
\begin{align*}
E_b f(t) = \text{e}^{j b t} f(t), \quad t \in \mathbb{R}.
\end{align*}

\item For $c > 0$, the operator $D_c$, called dilation by $c$, is defined by
\begin{align*}
D_c f(t) = \dfrac{1}{\sqrt{c}} f\left( \dfrac{t}{c} \right), \quad t \in \mathbb{R}.
\end{align*}
\end{enumerate}
\end{definition}

It should be noted that $\mathcal{L}^1$ is not a subspace of $\mathcal{L}^2$ and vice versa. However, these facts are useful \cite{page 205, FAA}:
\begin{enumerate}
\item If $f \in \mathcal{L}^1$ and $f$ is bounded, then $f \in \mathcal{L}^2$ since:
\begin{align*}
|f| \leq M \Rightarrow |f|^2 \leq M|f| \Rightarrow \int_{-\infty}^\infty |f(t)|^2 dt \leq \int_{-\infty}^\infty M|f(t)| dt < \infty.
\end{align*}

\item If $f \in \mathcal{L}^2$ and vanishes outside a finite interval $[a,b]$, then $f \in \mathcal{L}^1$ since:
\begin{align*}
\int_{-\infty}^\infty |f(t)| dt = \int_a^b |f(t)| dt \leq (b - a)^{1/2} \left( \int_a^b |f(t)|^2 dt \right)^{1/2} < \infty.
\end{align*}

This follows from the Cauchy-Scwharz inequality stated below \cite{page 118, FSE2010}.
\end{enumerate}

\begin{theorem}[Cauchy-Schwarz' inequality]
For all $f,g \in \mathcal{L}^2(\mathbb{R})$,
\begin{align*}
\left| \int_{-\infty}^\infty f(t) \overline{g(t)} dt \right| \leq \left( \int_{-\infty}^\infty |f(t)|^2 dt \right)^{1/2} \left( \int_{-\infty}^\infty |g(t)|^2 dt \right)^{1/2}.
\end{align*}
\end{theorem}

A definition similar to definition \ref{def:Lpspace} is given for $\ell^p$, which deals with sequences:
\begin{definition}[The $\ell^p$ spaces]
For any $p \in [1,\infty)$, the normed vector space $\ell^p(\mathbb{Z})$ is the subspace of $\mathbb{C}^N$ consisting of vectors with finite $\ell^p$ norm:
\begin{align*}
\|x\|_p = \left( \sum_{n\in\mathbb{Z}} |x[n]|^p \right)^{1/p}.
\end{align*}

$\ell^2(\mathbb{Z})$ is the space of square-summable sequences. The inner product is defined as:
\begin{align*}
\langle x,y \rangle = \sum_{n\in\mathbb{Z}} x[n] \overline{y[n]},
\end{align*}

and the norm is defined as:
\begin{align*}
\|x\| = \left( \sum_{n\in\mathbb{Z}} |x[n]|^2 \right)^{1/2}.
\end{align*}

Furthermore, the $\ell^\infty$ norm is defined as:
\begin{align*}
\|x\|_\infty = \sup_{n\in\mathbb{Z}}|x[n]|.
\end{align*}
\end{definition}