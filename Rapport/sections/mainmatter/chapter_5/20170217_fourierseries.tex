\subsection{Fourier series}
Functions can be represented as a linear combination of functions on the form $\sin nt$.
Or in other words linear combination of a oscillating function, sines, cosines or equivalently complex exponential.
\\\\
It can be an advantage to decompose functions, as a series or integral of "simpler" functions, to retrieve information.
\begin{align*}
	f(x) = \sum_n=1^\infty c_n f_n(x)\text{, or } f(x)= \int_{-\infty}^\infty g(u) h(u,x) du
\end{align*}
Information from a function $f(x)$ will, as an example be stored in the coefficient $\{c_n\}_{n=1}^\infty$.
The coefficients can easily be stored on a computer.
\\\\


A function $f: \mathbb{R}\to\mathbb{R}$ (or $\mathbb{C}$) is called $2\pi$-periodic, if $f(\theta + 2\pi) = f(\theta), \forall\theta\in\mathbb{R}$

\begin{align*}
	x(t) &= a_0 + \sum_{n=1}^\infty(a_n \cos(n \omega t) + b_n \sin(n \omega t))\\
	&= \sum_{n=-\infty}^{\infty} c_n e^{n j\omega t} 
\end{align*}
The second part holds per Euler's formula gives:
\begin{align*}
	\cos(n(\theta) = \dfrac{e^{j n \theta} + e^{-j n \theta}}{2} \text{ and } \sin(n \theta) = \dfrac{e^{jn\theta-}e^{jn\theta}}{2j}
\end{align*}

It can be simpler to work with the complex exponential function, but working with the trigonometric functions $\cos$ and $\sin$ have there advantages. 
Like being real-valued and odd and even, cosine and sine respectively.

\subsection{Calculation of the coefficient $c_n$}
It is a necessity to calculate the $c_n$ coefficient.
By the assumption that this can be done, and it's permissible to integrate the fourier series term by term: :\\
\\
To solve $c_n$ in the equation $f(t)= \sum_{n=-\infty}^{\infty} c_n e^{n j\omega t}$
\\\\
First multiply both sides by $e^{-(j\omega k t)}$, where $k\in \mathbb{Z}$. And integrate both sides over a given period, from $-\pi$ to $\pi$:

\begin{align*}
	\int_{-\pi}^\pi f(t)e^{-(j\omega k t)} = \int_{-\pi}^\pi \sum_{n=-\infty}^{\infty} c_n e^{n j\omega t} e^{-j\omega k t} dt
\end{align*}

The integration of a $2\pi$ - periodic function over the length of the period, can be shifted with an integer.

\begin{lemma}
Suppose $F$ is $2\pi$ - periodic and integrable. Then for any real number a 

\begin{align}
\int_a^{2\pi+a}F(t) dt = \int_0^{2\pi}F(t)dt
\end{align}
\end{lemma}
\begin{proof}
\begin{align*}
	\int_0^{2\pi}F(t)dt 
	&= \int_0^a F(t) dt + \int_a^{2\pi} 
	= \int_0^a F(t+2\pi)dt + \int_a^{2\pi} F(t) dt\\ 
	&= \int_{2\pi}^{2\pi + a} F(t) dt + \int_a^{2\pi}F(t)dt
	= \int_a^{2\pi}F(t)dt	+ \int_{2\pi}^{2\pi + a} F(t)dt \\
	&= \int_a^{2\pi+a}F(t)dt
\end{align*}
\end{proof}

Pulling the summation and the constant on the right-hand side out of the integral yields the following:

\begin{align*}
	\int_{-\pi}^\pi f(t) e^{-j \omega k t}dt
	= \sum_{-\infty}^\infty c_n \int_{\pi}^\pi e^{-j \omega(n-k)t}dt
\end{align*} 

The integral on the right-hand side has two cases to consider, $n \neq k$ and $n = k$\\\\

For $n\neq k$:
\begin{align*}
	\forall n,n\neq k: \int_{-\pi}^\pi e^{j\omega(n-k)t}dt 
	=\dfrac{1}{j\omega(n-k)}e^{j\omega(n-k)t}\mid_{-\pi}^{\pi}
	=\dfrac{(-1)^{n-k}-1(-1)^{n-k}}{j\omega(n-k)}
	=0
\end{align*}
\chr{Could be written with sine and cosine, to show that the integral of those 2 over a periode is 0}

The second case is for $n = k$
\begin{align*}
	\forall n,n=k: \int_{-\pi}^\pi e^{j\omega(n-k)t}dt = \int_{-\pi}^\pi dt = 2\pi
\end{align*}

The integral can be concluded to yield either $0$ or $2\pi$:

\begin{align}
	\int_{-\pi}^{\pi} e^{j \omega (n-k)t}dt 
	= 
	\begin{cases}
			2\pi \text{ if } n=k\\
			0 \text{ otherwise}
	\end{cases}
\end{align}

The coefficient $c_n$'s general equation is then:
\begin{align*}
	c_n = \dfrac{1}{2\pi} \int_{-\pi}^{\pi} f(t) e^{-(j \omega nt)}dt
\end{align*} 

\begin{definition}
Fourier series of a $2\pi$ - periodic integrabel function $f(t)$ is defined as:
\begin{align*}
	\sum_{n=-\infty}^\infty c_n e^{j n \omega t}
\end{align*}

where $c_n = \dfrac{1}{2\pi}\int_{- \pi}^\pi f(t) e^{-j n \omega t}$
\\\\
Alternatively:
\begin{align*}
	\dfrac{a_0}{2} + \sum_{n=1}^{\infty} \left[ a_n \cos(n \omega t) + b_n \sin(n \omega t)\right]
\end{align*} 
where
\begin{align*}
	a_n 
	&= \dfrac{1}{\pi} \int_{-\pi}^\pi f(t) \cos (n \omega t) dt, \, n=0,1,\dots\\
	b_n
	&= \dfrac{1}{\pi} \int_{-\pi}^\pi f(t) \sin (n \omega t) dt, \, n=1,2,\dots	
\end{align*}
\end{definition}

Further if a function $g(t)$ is uneven, ($g(t) = -g(-t)$), then the integral $\int_{-\pi}^\pi g(t) = 0$.

\subsection{Convergence of the Fourier series}

%http://cnx.org/contents/8YnJdzjg@8/Derivation-of-Fourier-Coeffici