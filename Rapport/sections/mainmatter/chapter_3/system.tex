In this chapter a model of and requirements to the system addressed in the problem statement is specified. This specification remains on a superficial level as a foundation to the system design in chapter \ref{ch7}. Furthermore test specifications of the system are made, by means of the used development model called the V-model. By this a delimitation of the project will be established. \\   
\\
Due to the presented problem statement the purpose of the application is basically to transform some music into a note sheet due to the sound. This is illustrated by figure \ref{fig:model1}.    
\begin{figure}[h]
\centering
\begin{tikzpicture}[auto, node distance=3.4cm, >=triangle 45, font=\sffamily ]
\draw
% Drawing the blocks of first filter :
	node at (0,0)[right=-15mm] {\textit{Music}}
	node [input, name=input1] {} 
	node [block, right of=input1] (app1) {?}
    node [output, name=output1, right of=app1]{}
    node at (6,0) [right=10mm]{\textit{Note sheet}}
;
% Joining blocks. 
% Commands \draw with options like [->] must be written individually
\draw[->](input1) -- node {Input}(app1);
\draw[->](app1) -- node {Output}(output1);

% Boxing and labelling noise shapers
\draw [color=gray,thick,dashed](-1.8,-1.2) rectangle (9.5,1.2);
\node at (-1.8,1.5) [above=5mm, right=0mm] {System};
\end{tikzpicture}
\caption{Basic concept of the system}
\label{fig:model1}
\end{figure}
In order for an application to perform this transformation an analogue sound has to be converted into a discrete time signal, in order for a computer to process the signal into the wanted outcome. \\
By this the system is fundamentally based upon theory about discrete time systems, including analysis and processing of such systems and the disposal data.

\section{Synthesizing}
To develop/design the system a synthesis approach is used to expand the basic concept of the system in order to define the purpose and thereby requirements to each part of the system.
Figure \ref{fig:synth} illustrates the synthesizing of the system down to a level showing the essential parts of the system on behave of the final purpose. The further synthesizing of the system are left for chapter \ref{ch7}\trine{will it be synthesizing i chaper 7?}. The next section elaborates the synthesizing process by defining the requirements for each part of the system.
\begin{figure}[H]
\centering

\tikzstyle{block} = [rectangle, draw,thick, 
    text width=4.2em, text centered, minimum height=3em]
    
%%%%% crazy opsætning til stor parentes %%%%%
\tikzset{round left paren/.style={ncbar=0.5cm,out=120,in=-120}}
\tikzset{round right paren/.style={ncbar=0.5cm,out=60,in=-60}}
\tikzset{
    ncbar angle/.initial=90,
    ncbar/.style={
        to path=(\tikztostart)
        -- ($(\tikztostart)!#1!\pgfkeysvalueof{/tikz/ncbar angle}:(\tikztotarget)$)
        -- ($(\tikztotarget)!($(\tikztostart)!#1!\pgfkeysvalueof{/tikz/ncbar angle}:(\tikztotarget)$)!\pgfkeysvalueof{/tikz/ncbar angle}:(\tikztostart)$)
        -- (\tikztotarget)
    },
    ncbar/.default=0.5cm,
}
%%%%%%%%%%%%%%%%%%%%%%%%%%%%%%%%%%%%%%%%%%%%%
\begin{tikzpicture}[auto, node distance=3cm, >=triangle 45, level 1/.style={sibling distance=70mm},
    level 2/.style={sibling distance=20mm},font=\sffamily ]
\draw
 % Drawing blocks 
	node at (0,0)[right=-15mm] {\textit{Music}}
	node [input, name=input1] {} 
	node [block, right of=input1] (app1) {?}
		child {
		node [block] (audio) {Audio file} 
			child {
				node [block] (mic) {Mic.}					
				}			
			child {
				node [block] (ADC) {ADC}					
				}
			child {
				node [block] (save) {Storage}
				}
			}
		child {
			node [block] (algorithm) {Algorithm}
			child {
				node [block] (four) {Signal mod.}
				}
			child {
				node [block] (filter) {Spectral analysis}
				}
			child {
				node [block] (spec) {Repr. of spectrum }
				}
			child {
				node [block] (symb) {Symbol process.}
				}	
			}
    node [output, name=output1, right of=app1]{}
    node at (output1) [right=1mm]{\textit{Note sheet}}
 % Drawing the blocks of 2. layer:  
;
% stor parantes 
\draw [blue, ultra thick] (8.6,-4) to [round left paren ] (8.6,-2);
\draw [blue, ultra thick] (10.4,-4) to [round right paren ] (10.4,-2);
% Joining blocks. 
% Commands \draw with options like [->] must be written individually
\draw[->](input1) -- node {Input}(app1);
\draw[->](app1) -- node {Output}(output1);

\end{tikzpicture}
\caption{Synthetic block-diagram of the system}
\label{fig:synth}
\end{figure}

        
\subsection{Requirements and delimitations}
\subsubsection{Audio file}
To process an analogue sound in a computer the sound has to be digitalized. This is done by an analogue to digital converter(ADC) which transform an audio signal from a microphone into digital bits. Further these bits has to be stored as a audio file format as MP3 format. This process are to be done by the sound card in the computer and some storage software. In order to reconstruct the analogue signal exact the software needs to provide a sampling frequency at 44,1 kHz and a bit resolution of 16 bits\cite{Mic}. Further about 5 MB memory are needed, which corresponds to a typically MP3 music file. \\
The audio files for this system are limited to consist of single tones, meaning no cords. The tones will be recorded in an anechoic chamber which implies control upon the amount of noise on the signal.\\      
The produced audio file will provide as the \textit{input} for the algorithm that process the sound. 

\subsubsection{Algorithm}
The main part of the system is the algorithm that will provide as an application which take an audio file as input and generate the corresponding note sheet. This involves modification of the signal, spectral analysis of the signal and further a representation of the signal spectrum.\\
\textbf{Signal modification} are needed in this project in order to select what frequencies are wanted for the signal analysis. This will be done by a digital filter designed to reduce additive noise on the signal. The filter will be linear and time invariant, thus the filter will be designed to a specific audio file.\trine{specify the type of filter later}             
\\ 
\textbf{Spectral analysis} is concerning the spectrum of frequencies corresponding to the signal. This involves transformation of the signal, from time- to frequency domain.     
The primary method used for this purpose is the Fourier transformation which will be used in this project, based on the learning goals for the project. Other method could be.. \trine{Alternativer til fourier?}. \\   
\textbf{Representation of frequency spectrum} 

 
Cf. chapter \ref{ch2} a note sheet is basically a spectrogram showing the frequency of the tones. The main purpose is thereby to identify the wanted frequencies on the audio file. This involves transformation of the signal, from time- to frequency domain done by Fourier Transformation. Filtering of the signal is necessary if any noise is included in the signal. The spectrogram is achieved by plotting the frequency. \\
To create the note sheet on behave of a spectrogram symbol processing is required, marked by the blue parentheses i figure \ref{fig:synth}. Based on the given time limit, and the theoretical focus in the project the system are delimited to leave out the step of symbol processing. Hence the spectrogram will make the last step in the algorithm. 

\subsubsection{Output} 
The result of the algorithm is as stated a spectrogram. A visualisation of the spectrogram will thereby make the output of the system. It is required that the spectrogram is how precise?       

\section{Final system}
On behave of the stated specifications to the system figure \ref{fig:system} illustrates the final system which will form a basis of the further work. Main focus of the project is the development of the algorithm in the application ..\trine{okay der står fourier og spectrogram?}   
\begin{figure}[H]
\centering
\tikzstyle{block} = [rectangle,draw,thick, text width=4.2em, text centered, minimum height=3em]
\begin{tikzpicture}[node distance=2.2cm,auto,font=\sffamily ,>=latex']
\draw
% Drawing the blocks of first filter :
	node at (0,0)[right=-13mm]{\huge\twonotes} % kan skiftet ud med sinus længere nede
	node [input, name=input1] {} 
	node at (input1) [above=5mm] {\textsf{Mic}}		
	node [block, right of=input1] (ADC) {ADC}
	node [block, right of=ADC] (store) {Storage}
	node [block, right of=store] (four) {Signal Mod.}
	node [block, right of=four] (filter) {Fourier Trans.}
	node [block, right of=filter] (spec) {Spectro-gram}
	node [block, right of=spec] (out) {Visual Spectrogram}
;
% Joining blocks. 
% Commands \draw with options like [->] must be written individually

\draw[->](input1) -- node {}(ADC);
\draw[->](ADC) -- node {}(store);
\draw[->](store) -- node {}(four);
\draw[->](four) -- node {}(filter);
\draw[->](filter) -- node {}(spec);
\draw[->](spec) -- node {}(out);
\filldraw[color=black,fill=white,thick](input1) circle (0.3);
\draw(-0.3,0.5) -- (-0.3,-0.5);
%sinus wave
%\draw (-1.5,0) sin (-1.40,0.5) cos (-1.30,0) sin (-1.20,-0.5) cos (-1.10,0) sin (-1,0.5) cos (-0.9,0) sin (-0.8,-0.5) cos (-0.7,0);

% Boxing and labelling noise shapers
\draw [color=gray,thick,dashed](-1.2,-1.3) rectangle (5.43,1.3);
\node at (-1.2,1.5) [above=5mm, right=0mm] {Input};
\draw [color=gray,thick,dashed](5.52,-1.3) rectangle (11.99,1.3);
\node at (5.55,1.5) [above=5mm, right=0mm] {Application};
\draw [color=gray,thick,dashed](12.085,-1.3) rectangle (14.3,1.3);
\node at (12.09,1.5) [above=5mm, right=0mm] {Output};
\end{tikzpicture}
\caption{Basic block-diagram illustrating the final system} 
\label{fig:system}
\end{figure}