In this chapter a model of and requirements to the system addressed in the problem statement are specified. This specification remains on a superficial level as a foundation to the system design in chapter \ref{ch10}. Furthermore, test specifications of the system are made by means of the development model called the V-model. Through this a delimitation of the project will be established. \\   
\\
Due to the presented problem statement the purpose of the system is to create a note sheet from input data representing music. This is illustrated in figure \ref{fig:model1}.    
\begin{figure}[H]
\centering
\begin{tikzpicture}[auto, node distance=3.4cm, >=triangle 45, font=\sffamily ]
\draw
% Drawing the blocks of first filter :
	node at (0,0)[right=-15mm] {\textit{Music}}
	node [input, name=input1] {} 
	node [block, right of=input1] (app1) {?}
    node [output, name=output1, right of=app1]{}
    node at (6,0) [right=10mm]{\textit{Note sheet}}
;
% Joining blocks. 
% Commands \draw with options like [->] must be written individually
\draw[->](input1) -- node {Input}(app1);
\draw[->](app1) -- node {Output}(output1);

% Boxing and labelling noise shapers
\draw [color=gray,thick,dashed](-1.8,-1.2) rectangle (9.5,1.2);
\node at (-1.8,1.5) [above=5mm, right=0mm] {System};
\end{tikzpicture}
\caption{Basic concept of the system.}
\label{fig:model1}
\end{figure}
In order for an application to perform this transformation an analog sound has to be converted into a discrete-time signal for a computer to process the signal into the wanted outcome. \\
By this the system is fundamentally based upon theory about discrete-time systems, including analysis and processing of such systems and the data at disposal.

\section{Synthesizing} \label{sec:synth}
To develop and design the system a synthesis approach is used to expand the basic concept of the application in order to define the purpose and thereby requirements for each part of the application.
Figure \ref{fig:synth} illustrates the synthesizing of the system down to a level showing the essential parts of the application. Requirements for each part of the application are elaborated in the next section.
\begin{figure}[H]
\centering

\tikzstyle{block} = [rectangle, draw,thick, 
    text width=4.2em, text centered, minimum height=3em]
    
%%%%% crazy opsætning til stor parentes %%%%%
\tikzset{round left paren/.style={ncbar=0.5cm,out=120,in=-120}}
\tikzset{round right paren/.style={ncbar=0.5cm,out=60,in=-60}}
\tikzset{
    ncbar angle/.initial=90,
    ncbar/.style={
        to path=(\tikztostart)
        -- ($(\tikztostart)!#1!\pgfkeysvalueof{/tikz/ncbar angle}:(\tikztotarget)$)
        -- ($(\tikztotarget)!($(\tikztostart)!#1!\pgfkeysvalueof{/tikz/ncbar angle}:(\tikztotarget)$)!\pgfkeysvalueof{/tikz/ncbar angle}:(\tikztostart)$)
        -- (\tikztotarget)
    },
    ncbar/.default=0.5cm,
}
%%%%%%%%%%%%%%%%%%%%%%%%%%%%%%%%%%%%%%%%%%%%%
\begin{tikzpicture}[auto, node distance=3cm, >=triangle 45, level 1/.style={sibling distance=70mm},
    level 2/.style={sibling distance=20mm},font=\sffamily ]
\node at (1.95,0.9) [above=5mm, right=0mm] {Application};
\draw
 % Drawing blocks 
	node at (0,0)[right=-15mm] {\textit{Music}}
	node [input, name=input1] {} 
	node [block, right of=input1] (app1) {?}
		child {
		node [block] (audio) {Audio file} 
			child {
				node [block] (mic) {Mic.}					
				}			
			child {
				node [block] (ADC) {ADC}					
				}
			child {
				node [block] (save) {Storage}
				}
			}
		child {
			node [block] (algorithm) {Algorithm}
			child {
				node [block] (four) {Signal mod.}
				}
			child {
				node [block] (filter) {Spectral analysis}
				}
			child {
				node [block] (spec) {Repr. of spectrum }
				}
			child {
				node [block] (symb) {Symbol process.}
				}	
			}
    node [output, name=output1, right of=app1]{}
    node at (output1) [right=1mm]{\textit{Note sheet}}
 % Drawing the blocks of 2. layer:  
;
% stor parantes 
\draw [blue, ultra thick] (8.6,-4) to [round left paren ] (8.6,-2);
\draw [blue, ultra thick] (10.4,-4) to [round right paren ] (10.4,-2);
% Joining blocks. 
% Commands \draw with options like [->] must be written individually
\draw[->](input1) -- node {Input}(app1);
\draw[->](app1) -- node {Output}(output1);

\end{tikzpicture}
\caption{Synthetic block diagram of the application.}
\label{fig:synth}
\end{figure}
        
\subsection{Requirements and delimitations}
\subsubsection{Audio file}
To process an analog sound in a computer the sound has to be digitalized. This is done by a Sample \& Hold circuit and an analog to digital converter (ADC), which transform an analog audio signal from a microphone into a digital signal discretized in time and amplitude. These bits are then to be stored as an audio file format as .WAV format. This process will be done by a sound card in a computer and storage software. In order to reconstruct the analog signal exactly the software needs to provide a sampling frequency of 44.1 kHz and a bit resolution of 16 bits \cite{Mic}. A typical .MP3 file needs about 5 MB of memory. \\
The audio files will contain music and noise separately recorded in an anechoic chamber. This minimizes unknown noise and allows the signal-to-noise ratio to be controlled. \\      
The produced audio file will act as the \textit{input signal} for the algorithm that processes the sound. Thus the system will not operate in real-time. 

\subsubsection{Algorithm}
The main part of the application is an algorithm, which takes an audio file as input and generates a corresponding note sheet as output. This involves signal modification, spectral analysis, a spectral representation and generation of a note sheet from the signal:
\begin{itemize}
\item[] \textbf{Signal modification} is needed in this project to select which frequencies are wanted for the spectral analysis and remove unwanted frequencies. The modification will be done by a digital filter designed to reduce noise in the signal - because noise is expected in a recorded sound signal. The filter will be linear and time invariant.\trine{behøves  LTI at nævnes her?} The choice of filter type and definition of desired specifications will be made on behalf of a frequency analysis of selected music and noise signals that could be representative for the use of the system.             
\item[] \textbf{Spectral analysis} concerns the spectrum of frequencies corresponding to the signal. This involves transformation of the signal from time domain to frequency domain.     
One primary method used for this purpose is the Fourier transformation. Based on the theoretical focus and learning goals for the project this transformation will be used in this project. An alternative to the Fourier transformation is the wavelet transform which will be briefly examined in the end of the project in order to compare it with the results of Fourier transformation.  
\item[] \textbf{Representation of frequency spectrum} will be done by plotting the frequency spectrum over time, which forms a spectrogram. By the spectrogram it is possible to analyse the frequencies of the filtered signal over time. As described in chapter \ref{ch2} a note sheet is basically a spectrogram showing the frequency of the tones. To compute a spectrogram the \textit{Short time Fourier transformation} (STFT) will be used whereby a trade of between accuracy in time and accuracy in frequency occurs.  

\item[] \textbf{Symbol processing} is required to create a note sheet on behalf of the spectrogram. 
However, based on the given time limit and the theoretical focus in the project, the system is delimited to leave out the step of actual symbol processing - marked by the blue parentheses in figure \ref{fig:synth}. As a simplified alternative a simple \textit{amplitude peak detection} algorithm will be made to ensure that the significant frequencies - which makes a tone or cord recognizable - can be detected from the spectrogram. \\       
Hence a visualization of the spectrogram and corresponding significant frequencies detected will make the last step in the algorithm, hence the \textit{output} of the system.
\end{itemize}

\section{Final system}
On behalf of the stated specifications to the system, figure \ref{fig:system} illustrates the final system which will form a basis of the further work. \\ \\
Focus in this project is the development of the algorithm hence the input of the algorithm is delimited to be a prerecorded audio file containing a know signal. \\
The algorithm inside the application is bounded to create a spectrogram illustrating the frequencies corresponding the tones on the audio file. The noise is expected to be reduced in such degree that it is possible to identify the significant frequencies by the spectrogram, which will form the output of the system. 
\begin{figure}[H]
\centering
\hspace*{-2cm}
\tikzstyle{block} = [rectangle,draw,thick, text width=4.2em, text centered, minimum height=3em]
\begin{tikzpicture}[node distance=2.2cm,auto,font=\sffamily ,>=latex']
\draw
% Drawing the blocks of first filter :
	node at (0,0)[right=-13mm]{\huge\twonotes} % kan skiftet ud med sinus længere nede
	node [input, name=input1] {} 
	node at (input1) [above=5mm] {\textsf{Mic}}		
	node [block, right of=input1] (ADC) {ADC}
	node [block, right of=ADC] (store) {Storage}
	node [block, right of=store] (four) {Filter}
	node [block, right of=four] (filter) {Fourier Trans.}
	node [block, right of=filter] (spec) {Spectro-gram}
	node [block, right of=spec] (peak) {Peak detection}
	node [block, right of=peak] (out) {Visualiza-tion}
;
% Joining blocks. 
% Commands \draw with options like [->] must be written individually

\draw[->](input1) -- node {}(ADC);
\draw[->](ADC) -- node {}(store);
\draw[->](store) -- node {}(four);
\draw[->](four) -- node {}(filter);
\draw[->](filter) -- node {}(spec);
\draw[->](spec) -- node {}(peak);
\draw[->](peak) -- node {}(out);
\filldraw[color=black,fill=white,thick](input1) circle (0.3);
\draw(-0.3,0.5) -- (-0.3,-0.5);
%sinus wave
%\draw (-1.5,0) sin (-1.40,0.5) cos (-1.30,0) sin (-1.20,-0.5) cos (-1.10,0) sin (-1,0.5) cos (-0.9,0) sin (-0.8,-0.5) cos (-0.7,0);

% Boxing and labelling noise shapers
\draw [color=gray,thick,dashed](-1.2,-1.3) rectangle (5.43,1.3);
\node at (-1.2,1.5) [above=5mm, right=0mm] {Input};
\draw [color=gray,thick,dashed](5.52,-1.3) rectangle (14.205,1.3);
\node at (5.55,1.5) [above=5mm, right=0mm] {Application};
\draw [color=gray,thick,dashed](14.3,-1.3) rectangle (16.515,1.3);
\node at (14.4,1.5) [above=5mm, right=0mm] {Output};
\end{tikzpicture}
\caption{Basic block-diagram illustrating the final system.} 
\label{fig:system}
\end{figure}