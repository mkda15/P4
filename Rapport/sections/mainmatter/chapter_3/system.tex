In this chapter a model of and requirements to the system is specified on a superficial level. These specifications forms a basis of the further work. 
\\ \\
Due to the presented problem statement and delimitation of the project the purpose of the application is basically to transform a sound into a spectrogram showing the frequency due to the sound. This is illustrated by figure \ref{fig:model1}.    
\begin{figure}[h]
\centering
\begin{tikzpicture}[auto, thick, node distance=3.4cm, >=triangle 45]
\draw
% Drawing the blocks of first filter :
	node at (0,0)[right=-15mm] {Sound}
	node [input, name=input1] {} 
	node [block, right of=input1] (app1) {Application}
    node [output, name=output1, right of=app1]{}
    node at (6,0) [right=10mm]{Spectrogram}
;
% Joining blocks. 
% Commands \draw with options like [->] must be written individually
\draw[->](input1) -- node {input}(app1);
\draw[->](app1) -- node {output}(output1);

% Boxing and labelling noise shapers
\draw [color=gray,thick,dashed](-1.8,-1.5) rectangle (9.5,1.5);
\node at (-1.8,1.7) [above=5mm, right=0mm] {\textsc{System}};
\end{tikzpicture}
\caption{Basic block-diagram of the system}
\label{fig:model1}
\end{figure}
In order for an application to perform this transformation an analogue sound has to be converted into a discrete time signal, in order for a computer to process the signal into the wanted outcome. \\
By this the system is fundamentally based upon theory about discrete time systems, including analysis and processing of such systems and the disposal data. \\
In the following sections specifications and requirements are specified for the three parts of the system.  

\section{Input}
To process an analogue sound in a computer the sound has to be digitalized, this is done by an analogue to digital converter(ADC). An ADC basically consistent of sampling of the signal done by a sample and hold unit(S/H) followed by quantifying of the samples.
        

\section{Output}

\section{Application}

