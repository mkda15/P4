\chapter{Discrete-time systems}\label{discrete_time_systems} \label{ch4}
This chapter will establish the theory of discrete-time systems needed for the discussion of the Fourier transform, sampling theory and filtering described in chapters \ref{ch6}, \ref{ch7}, and \ref{ch8}, respectively. The chapter is inspired by \cite{page 195-198, FSP}.
\\ \\
Firstly, discrete-time systems are defined in definition \ref{def:discrete_time_system}.

\begin{definition}[Discrete-time system]\label{def:discrete_time_system}
A discrete-time system is an operator $T: V \to V$ that maps a sequence $x\in V$ to a sequence $y\in V$ such that
\begin{align}
y=T(x).
\end{align}
\end{definition}
\frede{Indsæt måske figur med operator på en vektor \texttrademark}
\martin{Skal ligningerne have numre, når vi ikke bruger dem? \textregistered}

\section{Properties and types of discrete-time systems}\label{sec:properties_DTS}
Discrete-time systems can have multiple properties that are important to consider when working with them. The following definitions will cover the need in this project.
\begin{definition}[Linear system]\label{def:linear_system}
A discrete-time system $T$ is called linear when, for any inputs $x$ and $y$ and any $\alpha,\beta\in\mathbb{C}$,
\begin{align}
T(\alpha x + \beta y) = \alpha T(x) + \beta T(x).
\end{align}
\end{definition}

\begin{definition}[Memoryless system]\label{def:memoryless_system}
A discrete-time system T is called memoryless when, for any $k\in\mathbb{Z}$ and any inputs $x$ and $x'$,
\begin{align}
\mathbf{1}_{\{k\}}x=\mathbf{1}_{\{k\}}x'\,\Rightarrow\,\mathbf{1}_{\{k\}}T(x)=\mathbf{1}_{\{k\}}T(x'),
\end{align}

where $\mathbf{1}_{\{k\}}$ is the indicator function such that $\mathbf{1}_{\{k\}}(x)=1$ for $x=k$ and $\mathbf{1}_{\{k\}}=0$ for $x\neq k$.
\end{definition}

As such the output memoryless systems depend solely on current input.
\begin{definition}[Causal system]\label{def:causal_system}
A discrete-time system $T$ is called causal when, for any $k\in\mathbb{Z}$ and inputs $x$ and $x'$,
\begin{align}
\mathbf{1}_{\{-\infty,\ldots,k\}}x=\mathbf{1}_{\{-\infty,\ldots,k\}}x'\,\Rightarrow\,\mathbf{1}_{\{-\infty,\ldots,k\}}T(x)=\mathbf{1}_{\{-\infty,\ldots,k\}}T(x').
\end{align}
\end{definition}

As such the output of a causal system depends solely on the current and previous inputs.

\begin{definition}[Time-invariant system]\label{def:time_invariant_system}
A discrete-time system $T$ is called time-invariant when, for any $k\in\mathbb{Z}$ and input $x$,
\begin{align}
y=T(x)\,\Rightarrow\,y'=T(x'),
\end{align}

where $x_n'=x_{n-k}$ and $y_n'=y_{n-k}$.
\end{definition}

As such a time-invariant system does not depend on the time at which it operates - a shift in time for inputs causes an equal shift in outputs.
\begin{definition}[BIBO-stable system]\label{def:BIBO_stable_system}
A discrete-time system $T$ is called bounded-input, bounded-output stable (BIBO-stable) when
\begin{align}
x\in\ell^{\infty}(\mathbb{Z})\,\Rightarrow\,y\in\ell^{\infty}(\mathbb{Z}),
\end{align}

where $\ell^{\infty}(\mathbb{Z})$ is the space of bounded sequences.\footnote{See chapter \ref{ch4}.}
\end{definition}

\subsection{Linear and time-invariant systems}
Some discrete-times systems can be described by \textit{linear difference equations}:
\begin{align}\label{eq:linear_diff_equation}
y_n=\sum_{k=-\infty}^{\infty}b_k^{(n)}x_{n-k}-\sum_{k=1}{\infty}a_k^{(n)}y_{n-k}
\end{align}
\eqref{eq:linear_diff_equation} relates the inputs and earlier outputs of a system to the current output in linear fashion. If the difference equation furthermore is with constant coefficients (not dependent on $n$) it becomes a linear \textit{time-invariant} difference equation as seen in \eqref{eq:LTI_diff_equation} and describes a \textit{linear time-invariant system} (LTI system).

\begin{align}\label{eq:LTI_diff_equation}
y_n=\sum_{k=-\infty}^{\infty}b_kx_{n-k}-\sum_{k=1}^{\infty}a_ky_{n-k}
\end{align}
\eqref{eq:LTI_diff_equation} is non-causal and infinite in input as the output $y_n$ depends on future values $x_n$ and infinitely many of these. To make the system causal and finite (and thus realizable) the number of coefficients are made finite reducing \eqref{eq:LTI_diff_equation} to
\begin{align}\label{eq:LTI_diff_equation_finite}
y_n=\sum_{k=0}^Mb_kx_{n-k}-\sum_{k=1}^Na_ky_{n-k}.
\end{align}
This project will focus on systems described by \eqref{eq:LTI_diff_equation_finite}.
\subsubsection{Impulse response}
An important characteristic of LTI systems is the unique specification of the system by its \textit{impulse response}.
\begin{definition}[Impulse response]\label{def:impulse_response}
A sequence h is called the impulse response of a discrete LTI system T when
\begin{align}
h=T(\delta),
\end{align}
where $\delta(t)=1$ for $t=0$ and $\delta(t)=0$ otherwise.
\end{definition}
The impulse response will be of importance in chapter \ref{ch7} when designing filters which are discrete-time systems.