\section{Properties and types of discrete-time systems}\label{sec:properties_DTS}
Discrete-time systems can have multiple properties that are important to consider when working with them. The following definitions will cover the need in this project.
\begin{definition}[Linear system]\label{def:linear_system}
A discrete-time system $T$ is called linear when, for any inputs $x[n]$ and $y[n]$ and any $\alpha,\beta\in\mathbb{C}$,
\begin{align}
T(\alpha x[n] + \beta y[n]) = \alpha T x[n] + \beta T x[n].
\end{align}
\end{definition}

\begin{definition}[Memoryless system]\label{def:memoryless_system}
A discrete-time system $T$ is called memoryless when, for any $k\in\mathbb{Z}$ and any inputs $x[n]$ and $\tilde{x}[n]$,
\begin{align}
\mathbf{1}_{\{k\}}x[n] = \mathbf{1}_{\{k\}}\tilde{x}[n] \, \Rightarrow \, \mathbf{1}_{\{k\}} T x[n] = \mathbf{1}_{\{k\}}T\tilde{x}[n],
\end{align}

where $\mathbf{1}_{\{k\}}$ is the indicator function such that $\mathbf{1}_{\{k\}}x[n]=x[n]$ for $n=k$ and $\mathbf{1}_{\{k\}}x[n]=0$ for $n \neq k$.
\end{definition}

Therefore, memoryless systems are instantaneous and depend solely on current input.

\begin{definition}[Causal system]\label{def:causal_system}
A discrete-time system $T$ is called causal when, for any $k\in\mathbb{Z}$ and inputs $x[n]$ and $\tilde{x}[n]$,
\begin{align}
\mathbf{1}_{\{-\infty,\ldots,k\}} x[n] = \mathbf{1}_{\{-\infty,\ldots,k\}} \tilde{x}[n] \, \Rightarrow \, \mathbf{1}_{\{-\infty,\ldots,k\}} Tx[n] = \mathbf{1}_{\{-\infty,\ldots,k\}} T\tilde{x}[n].
\end{align}
\end{definition}

As such the output of a causal system at time index $k$ depends solely on the inputs up to time index $k$.

\begin{definition}[Time-invariant system]\label{def:time_invariant_system}
A discrete-time system $T$ is called time-invariant when, for any $k\in\mathbb{Z}$ and input $x[n]$,
\begin{align}
y = Tx[n] \, \Rightarrow \, \tilde{y} = T\tilde{x}[n],
\end{align}

where $\tilde{x}[n] = x[n-k]$ and $\tilde{y}[n] = y[n-k]$.
\end{definition}

As such a time-invariant system does not depend on the time at which it operates; a shift in time for inputs causes an equal shift in outputs.

\begin{definition}[BIBO-stable system]\label{def:BIBO_stable_system}
A discrete-time system $T$ is called bounded-input, bounded-output stable (BIBO-stable) when
\begin{align}
x[n]\in\ell^{\infty}(\mathbb{Z})\,\Rightarrow\,y[n]\in\ell^{\infty}(\mathbb{Z}),
\end{align}

where $\ell^{\infty}(\mathbb{Z})$ is the space of bounded sequences.$^[$\footnote{See chapter \ref{ch5}.}$^]$
\end{definition}

\section{Linear and time-invariant systems}\label{sec:LTI}
Some discrete-times systems can be described by \textit{linear difference equations} \cite{page 202, FSP}:
\begin{align} \label{eq:linear_diff_equation}
y[n] = \sum_{k=-\infty}^{\infty} b^{(n)} [k] x[n-k] - \sum_{k=1}^{\infty} a^{(n)} [k] y[n-k],
\end{align}

where $b^{(n)}$ is the $n$'th derivative of $b$. \eqref{eq:linear_diff_equation} relates the inputs and earlier outputs of a system to the current output in linear fashion. If the difference equation furthermore is with constant coefficients (not dependent on $n$) it becomes a linear \textit{time-invariant} difference equation and describes a \textit{linear time-invariant system} (LTI system):
\begin{align}\label{eq:LTI_diff_equation}
y[n] = \sum_{k=-\infty}^{\infty} b[k] x[n-k] - \sum_{k=1}^{\infty} a[k] y[n-k].
\end{align}

\eqref{eq:LTI_diff_equation} is non-causal and infinite in input as the output $y[n]$ depends on infinitely many future values of $x[n]$. To make the system causal and finite (and thus realizable) the number of coefficients are made finite by reducing \eqref{eq:LTI_diff_equation} to
\begin{align}\label{eq:LTI_diff_equation_finite}
y[n] = \sum_{k=0}^M b[k] x[n-k] - \sum_{k=1}^N a[k] y[n-k].
\end{align}

This project will focus on systems described by \eqref{eq:LTI_diff_equation_finite}.

\subsection{Impulse response}
An important characteristic of LTI systems is the unique specification of the system by its \textit{impulse response}.
\begin{definition}[Impulse response]\label{def:impulse_response}
A sequence h is called the impulse response of a discrete LTI system T when
\begin{align}
h[n]=T\delta[n],
\end{align}

where $\delta[n] = 1$ for $n = 0$ and $\delta[n] = 0$ otherwise.
\end{definition}

Therefore, the impulse response $h$ of a causal linear system always satisfies that $h[n] = 0$ for all $n < 0$. An impulse response of the LTI system described by \eqref{eq:LTI_diff_equation_finite} is an output that results from the input $x[n] = \delta[n]$:
\begin{align*}
h[n] = \sum_{k=0}^M b[k] \delta[n-k] - \sum_{k=1}^N a[k] h[n-k] = b[n] - \sum_{k=1}^N a[k] h[n-k],
\end{align*}

where the first sum reduces to $b[n]$ for $n = k$.
\subsection{Convolution}\label{sec:convol}
The concept of convolution is of importance when dealing with discrete-time systems and impulse responses and will be defined in this section.\\\\
Writing $x[n]$ in \eqref{eq:DTS} as a sum by using the definition of the impulse response and by linearity, it follows that:
\begin{align}
y[n]&=Tx[n]
=T\sum_{k\in\mathbb{Z}}x[n]\delta[n-k]
=\sum_{k\in\mathbb{Z}}x[n]T\delta[n-k]\nonumber\\
&=\sum_{k\in\mathbb{Z}}x[n]h[n-k].\label{eq:convolution}
\end{align}
\eqref{eq:convolution} is defined as a convolution sum.

\begin{definition}[Convolution sum] \label{def:convolution}
The linear convolution between sequences $h[n]$ and $x[n]$ is defined as
\begin{equation}
(h*x)[n]=\sum_{k\in\mathbb{Z}}x[n]h[n-k]=\sum_{k\in\mathbb{Z}}x[n-k]h[k],
\end{equation}
where $n$ specifies the argument of the convolution sum.
\end{definition}

This means that, given an impulse response for a discrete LTI system, the output to a specific input can be calculated by convoluting the input with the impulse response. An impulse response is often called a filter, and convoluting an input with the impulse response is called filtering. The convolution is of importance in chapter \ref{ch8}, which covers filter theory and design in depth. The convolution is furthermore of importance in chapter \ref{ch6}, where the Fourier transform is discussed.