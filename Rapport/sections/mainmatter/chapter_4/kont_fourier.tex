\section{The Fourier transform}
This section deals with the continuous Fourier transform. In this project the Fourier transform will be used to transform a signal $f(t)$ represented in the time domain to a signal represented in the frequency domain, which will be useful when making the spectrograms mentioned in chapter \ref{ch1} \martin{Introducerer vi spektrogrammerne i kapitel 1? \textregistered}. The section is inspired by \cite{FourierTrans} and \cite{FAA}.
\\ \\
The Fourier transform is a generalization of the Fourier series described in the former section, where $f(t) = \sum_{n=-\infty}^\infty c_n e^{\frac{2n\pi j t}{T}}$. This sequence may be regarded as a function of $n$, and $f(t)$ and $c_n$ are thus different representations of the same object. $c_n$ may also be regarded as a function of the frequency $\omega = \frac{2n\pi}{T}$. If the period $T$ is large, then $\omega$ becomes small, and then $Tc_n$ becomes a continuous function for $T \to \infty$. Furthermore, for $T \to \infty$ it is no longer required that $f(t)$ is periodic since the period becomes $(-\infty,\infty)$. Therefore, the Fourier transform $F(\omega)$ of $f(t)$ is the limit of the continuous function $\frac{1}{\sqrt{2\pi}} T c_n$ for $T \to \infty$, which means that:
\begin{align*}
\dfrac{1}{\sqrt{2\pi}} T c_n = \dfrac{1}{\sqrt{2\pi}} \int_{-T/2}^{T/2} f(t) &e^{\frac{-2n\pi j t}{T}} dt = \dfrac{1}{\sqrt{2\pi}} \int_{-T/2}^{T/2} f(t) e^{-j\omega t} dt \Rightarrow \\
\dfrac{1}{\sqrt{2\pi}} T c_n &\to \dfrac{1}{\sqrt{2\pi}} \int_{-\infty}^\infty f(t) e^{-j \omega t} dt
\end{align*}

\noindent
For the Fourier series expansion of $f(t)$ with $\omega = \frac{2n\pi}{T}$, let 2 consecutive values of $n$ be 1 length apart so that $dn = 1$. This means that $d\omega = \frac{2\pi}{T} dn = \frac{2\pi}{T}$ and $\frac{T d\omega}{2\pi} = 1$, which shows that:
\begin{align*}
f(t) &= \sum_{n=-\infty}^\infty c_n e^{\frac{2n\pi j t}{T}} = \sum_{\frac{T\omega}{2\pi}=-\infty}^\infty c_n e^{j \omega t} \dfrac{T d\omega}{2 \pi} \\
&= \dfrac{1}{\sqrt{2\pi}} \sum_{\frac{T\omega}{2\pi}=-\infty}^\infty \dfrac{T c_n}{\sqrt{2\pi}} e^{j \omega t} d\omega \to \dfrac{1}{\sqrt{2\pi}} \int_{-\infty}^\infty F(\omega) e^{j \omega t} d \omega
\end{align*}

\noindent
Notice that $\frac{1}{2\pi}$ in the first line equals $\frac{1}{\sqrt{2\pi}} \cdot \frac{1}{\sqrt{2\pi}}$ in the second line, and this substitution makes both of the transforms (stated below) symmetric and also certain calculations easier. \cite{Page 10, FourierTrans}
\\ \\
\begin{definition}{The Fourier transform}
\\
For a continuous differentiable function $f(t)$ the Fourier transform is defined as follows:
\begin{align*}
F(\omega) = \dfrac{1}{\sqrt{2\pi}} \int_{-\infty}^\infty f(t) e^{-j \omega t}
\end{align*}
\end{definition}

\begin{definition}{The inverse Fourier transform} \label{InverseFourier}
\\
The inverse Fourier transform of $F(\omega)$ is:
\begin{align*}
f(t) = \frac{1}{\sqrt{2\pi}} \int_{-\infty}^\infty F(\omega) e^{j \omega t} dt
\end{align*}
\end{definition}

As stated in \cite{FourierTrans}, the mathematical importance of the Fourier transform is:
\begin{enumerate}
\item If $f(t)$ has properties that are not desirable (such as discontinuities, non smoothness), then $F(\omega)$ is possibly better behaved.
\item The Fourier transform represents a function that is not necessarily periodic and is defined on an infinite interval. The only requirement for the Fourier transform to exist is that the integral $\int_{-\infty}^\infty |f(t)| dt$ is convergent. The convergence of the Fourier transform is examined in chapter \ref{ch5}.
\\ \\
The Fourier transform $F(\omega)$ is a function of frequency but it also provides information on the amplitude and the phase of the signal at various frequencies. $F(\omega)$ may be written in polar coordinates as $F = |F|e^{j\theta}$ with the modulus $|F|$ representing the amplitude of the signal at a certain frequency $\omega$ and the phase shift $\theta = \arctan(\frac{Im(F)}{Re(F)}$ at the frequency $\omega$.
\end{enumerate}

\begin{definition}{Convolution}
\\
Let $f,g$ be functions defined on $\mathbb{R}$. Their convolution is the function $f*g$ defined by:
\begin{align*}
(f*g)(t) = \int_{-\infty}^\infty f(t-\tau) g(\tau) d\tau
\end{align*}
Provided that the integral exists.
(similar definition for the discrete case).
\end{definition}

\noindent
The following statement shows the symmetry between $f(t)$ and $F(\omega)$ and is inspired by \cite{FAA}.

\begin{theorem}
Let $f \in \mathcal{L}^1$. Then:

\begin{enumerate}[label=(\alph*)]
\item For any $a \in \mathbb{R}$:
\begin{align*}
\mathcal{F}\{f(t-a)\}(\omega) = e^{-ja\omega} F(\omega) \\
\mathcal{F}\{e^{jat}f(t)\}(\omega) = F(\omega - a)
\end{align*}

\item If $\delta > 0$ and $f_\delta(t)=\delta^{-1}f(t/\delta)$, then
\begin{align*}
\mathcal{F}\{f_\delta\}(\omega) = F(\delta\omega) \\
\mathcal{F}\{f(\delta t)\} = F_\delta(\omega)
\end{align*}

\item If $f$ is continuous and piecewise smooth and $f' \in L^1$, then
\begin{align*}
F'(\omega) = j\omega F(\omega)
\end{align*}

On the other hand, if $tf(t)$ is integrable, then
\begin{align*}
\mathcal{F}\{tf(t)\} = j F'(\omega)
\end{align*}

\item If also $g \in L^1$, then
\begin{align*}
\mathcal{F}\{f*g\}(\omega) = F(\omega) \cdot G(\omega)
\end{align*}
\end{enumerate}
\end{theorem}

\begin{proof}
\begin{enumerate}[label=(\alph*)]
\item
For the first equation of (a):
\begin{align*}
\mathcal{F}\{f(t-a)\}(\omega) = \int_{-\infty}^\infty e^{-j\omega t} f(t - a) dt = \int_{-\infty}^\infty e^{-j\omega t - j\omega a} f(t) dt = e^{-ja \omega} F(\omega)
\end{align*}

And for the second equation of (a):
\begin{align*}
\mathcal{F}\{e^{jat} f(t)\}(\omega) = \int_{-\infty}^\infty f(t) e^{jat} e^{-j \omega t} = \int_{-\infty}^\infty f(t) e^{j(a- \omega)t} = F(\omega - a)
\end{align*}

\item For (b):
\begin{align*}
\dots
\end{align*}

\item Since $f' \in L^1$ the limit
\begin{align*}
\lim_{t \to infty} f(t) = f(0) + \int_0^\infty f'(t) dt
\end{align*}

exists, and it must be zero since $f \in L^1$ as well. Likewise, $\lim_{t \to -\infty} = 0$, too. Integration by terms yields:
\begin{align*}
F'(\omega) = \int e^{-j \omega t} f'(t) dt = - \int (-j \omega) e^{-j \omega t} f(t) dt = j\omega F(\omega)
\end{align*}

If $tf(t)$ is integrable then:
\begin{align*}
\mathcal{F}\{tf(t)\} = \int e^{-j \omega t} t f(t) dt = j \dfrac{d}{d\omega} \int e^{-j \omega t} f(t) dt = j F'(\omega)
\end{align*}

Note that $t e^{-j \omega t} = j \dfrac{d}{d\omega} e^{-j\omega t}$.

\item Finally:
\begin{align*}
\mathcal{F}\{f*g\}(\omega) &= \iint e^{-j \omega t} f(t - y) g(y) dy dx \\
&= \iint e^{-j\omega(t-y)} f(t-y) e^{-j\omega y} g(y) dx dy \\
&= \iint e^{-j\omega z} f(z) e^{-j\omega y} g(y) dz dy \\
&= F(\omega) \cdot G(\omega)
\end{align*}

Where $z = t - y$.
\end{enumerate}
\end{proof}

\noindent
In all simplicity, part (a) of this theorem shows that translating a function (that is, a delay) corresponds to multiplying its Fourier transform by an exponential (which is equivalent to a phase shift since $F$ may be written in the polar coordinates $F = |F|e^{j\theta}$) and vice versa; part (b) shows that dilating a function by $\delta$ corresponds to dilating its Fourier transform by $1/\delta$ and vice versa; part (c) shows that differentiating a function corresponds to multiplying its Fourier transform by the coordinate variable and vice versa; and by the inverse Fourier tranform stated in definition \ref{InverseFourier}, it follows from part (d) that $F*G = 2 \pi (F \cdot G)$ \cite{FAA}.

Contents:
\begin{itemize}
\item The Fourier transform on $\mathcal{L}^1$
\item The inverse Fourier transform
\item The Fourier transform on $\mathcal{L}^2$
\item Fourier Transform theorems (page 60, Schafer)
\end{itemize}