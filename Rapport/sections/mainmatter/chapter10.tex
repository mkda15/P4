\chapter{Frequency analysis and filter design}
In this chapter the results form the recordings in the previous chapter will undergo frequency analysis. This will detect the frequency composition of the music and noise signals which aids the design of filters to filter out noise. The frequency analysis is furthermore necessary for the creation of spectrograms illustrating the frequency composition of the signals in time.\\\\
The frequency analysis is made in Python with the fast Fourier transform algorithm included in the numpy package.
\section{Frequency analysis of music}
In this section the recordings of music will undergo frequency analysis the goal of which is to be ableto express which frequencies music is generally found at.
\subsection{Single tone}
Firstly, the tuning of guitar is checked for consistency. The low and high E strings on the guitar should vibrate and emit sounds frequencies of 82.41 Hz and 329.63 Hz respectively. In figures (* insert references *) are seen the frequency spectra for the recordings of the two tones.
\begin{center}
* Insert plots of frequency spectra *
\end{center}
The most significant frequencies in the two recordings are 163.82 Hz and 329.83 for the low and high E's respectively. As $163.82$ Hz $\approx 2\cdot82.41$ Hz this is regarded as an overtone of the low E. The overtones are moreover observable in the figures as reduced peaks at integer multiples of the fundamental frequencies of the tones. It is furthermore seen from the figures, that the energy in the signals is mainly located at frequencies above 75 Hz and below 1000 Hz for the low E and above 100 Hz and below 2000 Hz for the high E.
\subsection{Chord}
In figure (* insert reference *) and figure (* insert reference *) are seen the frequency spectra of the recordings of low an high E chords respectively.
\begin{center}
* Insert figures *
\end{center}
The most significant frequencies are 163.86 Hz and 119.27 Hz. Once again it is assumed, that the higher frequency of the low pitch E is from overtones. These frequencies do furthermore not correspond to a specific note - this is assumed to be because of the composition of chords being of multiple tones. The majority of the energy in the signals is located above 80 Hz and below 2000 Hz.
\subsection{Scale}
In figure (* insert references *) is seen the frequency spectrum of playing a octatonic scale.
\begin{center}
* Insert plots *
\end{center}
The majority of the energi in the signal is as seen located above 100 Hz and below 600 Hz.
\subsection{Melody with single notes}
In figures (* insert reference *) is seen the frequency spectrum for a melody consisting only of single tones played slowly.
\begin{center}
* Insert plots *
\end{center}
The majority of energy in the signal is located above 100 Hz and below 2000 Hz.
\subsection{Melody with chords}
In figures (* insert reference *) is seen the frequency spectrum for a melody consisting only of single tones played slowly.
\begin{center}
* Insert plots *
\end{center}
The majority of energy in the signal is located above 90 Hz and below 800 Hz.
\section{Frequency analysis of noise}



\clearpage
Dette kapitel skal indeholde
\begin{itemize}
\item Frekvensanalyse af signaler herunder
\begin{itemize}
\item Amplitudespektrum for signalerne (måske i bilag)
\item Maksimale amplituder og tilhørende frekvenser
\item Eventuel genkendelse af overtoner
\end{itemize}
\item Generelt indtryk af amplitudespektra for musik kontra støj - forskelle/ligheder
\item Filterdesign på baggrund af ovenstående
\end{itemize}