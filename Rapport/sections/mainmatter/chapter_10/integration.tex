This section documents the integration of each unit into one algorithm that makes the application of the final system. The integration is validated according to the test specifications, section \ref{sec:testspec}, by feeding each output into the next unit and checking the results. 

\subsection{Implementation}
Each unit is integrated into one algorithm on behalf of the system specification in chapter \ref{ch3}. Algorithm \ref{alg:final} show the basic implementation of the final application. Note that each unit are imported as a functions to be called and fed with their respective inputs in the final script, as showed in line 1. Further from the Scipy-package \textit{scipy.io.wavfile} is importet as \textit{siw} 
\begin{algorithm}[H]
\caption{STFT algorithm}
\label{STFTalg}
\begin{algorithmic}[1]
\State \textbf{import functions} FFT, Filter, STFT, Spectrogram, Peak\_dec  
\\
\Procedure{Import datafiles}{}
\State \textbf{Return} signal, freq \Comment{List of signal, sampling frequency}
\EndProcedure 
\\
\State $sampels = len(signal)$ 
\State $time   = sampels/float(freq)$
\State $cut1 = f_{c1}/freq$ \Comment {Filter specifications}
\State $cut2 = f_{c2}/freq$
\State $delta = \delta$
\State $tw = f_{tw}/freq$
\\
\Procedure  {Filtering of signal in frequency domain}{}
\State $SIGNAL = FFT(signal) $
\State $H = FFT(Filter(cut1,cut2,freq))$
\State $filt\_SIGNAL = SIGNAL \cdot H$
\State $filt\_signal = IFFT(filt\_SIGNAL)$ \Comment {Filtered signal in time domain} 
\EndProcedure 
\\
\Procedure {Generate spectrogram and detect peaks}{} 
\State $X = STFT(filt\_signal)$
\State $spec = Spectrogram(X,time,freq)$
\State $peaks = Peak\_dec(?,?)$
\EndProcedure
\\
\Procedure {Generate output}{}
\State $plot(spec)$ 
\State $plot(peaks)$
\EndProcedure
\end{algorithmic}
\label{alg:final}
\end{algorithm}      

\subsection{Integration test}
