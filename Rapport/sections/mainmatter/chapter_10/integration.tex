This section documents the integration of each unit into one algorithm that makes the application of the final system. The integration is tested as specified in section \ref{sec:testspec} by feeding each output into the next unit and checking the results. 

\subsection{Implementation}
Each unit is integrated into one algorithm based on the system specifications described in chapter \ref{ch3}. Algorithm \ref{alg:final} shows the basic implementation of the final application. Note that each unit is imported as a function to be called and fed with their respective inputs in the final script as shown in line 1.
\begin{algorithm}[H]
\caption{Final algorithm}
\begin{algorithmic}[1]
\State \textbf{import functions} FFT, Filter, STFT, Spectrogram, Peak\_dec  
\\
\Procedure{Import datafiles}{}
\State $data = filename.wav$ \Comment{Import of signal}
\EndProcedure 
\\
\State $samples =$ length of signal
\State $time   = samples/$float$(f_s)$
\\
\Procedure  {Filtering of signal in frequency domain}{}
\State $SIGNAL = FFT(signal) $
\State $H = FFT(Filter(f_{c1},f_{c2},f_s))$
\State $filt\_SIGNAL = SIGNAL \cdot H$
\State $filt\_signal = \ $np.ifft$(filt\_SIGNAL)$
\EndProcedure 
\\
\Procedure {Generate spectrogram and detect peaks}{} 
\State $X = STFT(FFTsize,overlap,filt\_signal)$
\State $x,y = Spectrogram(X,f_s)$
\State $peaks = Peak\_dec(X)$
\EndProcedure
\\
\Procedure {Generate output}{}
\State $spec=pcolormesh(x,y,X)$ 
\State $clb = colorbar(spec)$
\State $plot(peaks)$
\EndProcedure
\end{algorithmic}
\label{alg:final}
\end{algorithm}          

\subsection{Integration test}
During the implementation the output of each step is printed and verified before implementing the next step. This is illustrated by figure \ref{fig:inte_validation}, where the input signal consists of an $E_2$ tone with noise of clapping hands.

\begin{figure}[H]
\centering
\begin{subfigure}{0.49\textwidth}
\centering
\includegraphics[width=\textwidth]{figures/validation/integration/signal.png}
\caption{Input signal.}
\label{fig:inte_signal}

\includegraphics[width=\textwidth]{figures/validation/integration/FSIGNAL.png}
\caption{Frequency spectrum of signal.}
\label{fig:inte_SIGNAL}

\includegraphics[width=\textwidth]{figures/validation/integration/spectrogram.png}
\caption{Spectrogram of filtered signal.}
\label{fig:inte_spec}

\end{subfigure}
\begin{subfigure}{0.49\textwidth}
\centering

\includegraphics[width=\textwidth]{figures/validation/integration/f_signal.png}
\caption{Filtered input signal.}
\label{fig:inte_signal_filt}

\includegraphics[width=\textwidth]{figures/validation/integration/f_FSIGNAL.png}
\caption{Frequency spectrum of filtered signal.}
\label{fig:inte_SIGNAL_filt}

\includegraphics[width=\textwidth]{figures/validation/integration/peak_dec.png}
\caption{Maximum frequencies over time.}
\label{fig:inte_peak}

\end{subfigure}
\caption{Results of integration of each unit into one application.}
\label{fig:inte_validation}
\end{figure}
Figures \ref{fig:inte_signal} and \ref{fig:inte_SIGNAL} show the input signal in the time and frequency domains, respectively. The FFT used is verified as the frequencies with the highest amplitudes are seen to correspond to the known frequencies of an $E_2$ tone (82.41 Hz) and two corresponding harmonics (164.82 Hz and 247.23 Hz). Furthermore, the noise appears similar to white noise across all frequencies.\\
Figures \ref{fig:inte_SIGNAL} and \ref{fig:inte_SIGNAL_filt} show the filtered signal in the time and frequency domains, respectively. Comparing this to the results of the unit test of the filter, the result is the same, and hence the filtering has been integrated as intended. Figure \ref{fig:inte_spec} shows a valid spectrogram computed from the filtered signal. Figure \ref{fig:inte_peak} shows the frequencies with the highest amplitudes over time, which is seen to match the hottest point in the spectrogram. \\
\\
From this it is verified that the implementation of each unit integrated into one single script works, which forms the final application. \\
It is concluded that the final applicationw works as intended corresponding to the theory used in the implementation. Whether this fulfils the practical purpose of the application when the input gets more complicated is tested in the next section.