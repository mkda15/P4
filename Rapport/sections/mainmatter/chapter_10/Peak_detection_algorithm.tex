Note location and Peak detection

To figure out which stand alone tone is being played in a given song/audio fill, to a given time, the STFT is analyzed, and the peak amplitude of the given time is located.\\
\\

%It can be expected that 
The frequency with the highest amplitude, and henceforth the largest amount of energy, is the most represented frequency. 
In an audio file, this will be the frequency of the tone being played or one of the overtones of this frequency \ref{ch2} \chr{ref chapter 2}.
This will hold as long as the signal to noise ratio of the signal isn't smaller than a given lower limit further explained in \ref{ch:snr?}. \chr{ref snr afsnit og er det efter Peak detection}
\\
\\
A visualization of this, is finding the peaks in the output spectrogram.
Due to Heisenbergs inequality \ref{Heis??} \chr{Heisenberg} a compromise between the time and frequency resolution has to be made and an approximation of the notes being played is a possibility.'

%\begin{algorithm}[H]
%\caption{Amplitude peak detection of short time Fourier transfrom}
%\label{alg:FIR}
%\begin{algorithmic}[1] 
%\Procedure{Compute peak amplitude}{}
%\State  Calculated STFT> $X$ \Comment {Given STFT calculation in an array $X$}
%\State $Y =$ linspace $0,$ Nyquist, length W \Comment Generate linear spaced vectors with as many points as in the window for the STFT.
%	\For {each integer $i$ in length of $X$}
%		\State $A = max(X[i])$ \Comment {A is set to the maximum amplitude for each FFT}
%		\State max\_freq\_pos = where in X[i], $(A = max(X[i]))$ \Comment {the location of the max amplitude is located}
%		\State max\_freq\_time = 
%	\EndFor
%	\State Return max\_freq\_time
%\EndProcedure
%\Procedure{Locate the associated frequency}
%\State A
%\EndProcedure
%\end{algorithmic}
%\end{algorithm}
%



With the signal STFT, each segmented data set corresponding to a FFT of at a given time 
The frequency in an audio file, 