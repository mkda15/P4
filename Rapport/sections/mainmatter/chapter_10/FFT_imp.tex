This section describes the implementation of the FFT discussed in section \ref{sec:FFT}. This implementation is assessed by comparing it to an implementation of the DFT discussed in section \ref{sec:DFT} and the FFT from the numpy-package in Python. The implementation of the DFT and FFT is described in the following algorithms.

\begin{algorithm}
\caption{DFT algorithm}
\label{DFTalg}
\begin{algorithmic}[1]
\Procedure{DFT function}
X = np.zeros(c,dtype=complex)
\EndProcedure
\end{algorithmic}
\end{algorithm}

\begin{algorithm}
\caption{FFT algorithm}
\label{FFTalg}
\begin{algorithmic}[1]
\State $N = $ \Comment{Number of samples to be computed}
\end{algorithmic}
\end{algorithm}