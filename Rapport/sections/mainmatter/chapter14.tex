\chapter{Future implications} \label{ch14}
This chapter considers future implications of the final system described in chapter \ref{ch10}. The Fourier transform is a rather old and well-known method of frequency analysis, and the STFT provides a fine method of representing and analysing a signal in both time and frequency. However, the relation between the resolutions of time and frequencies in the STFT is lower bounded due to Heisenberg's uncertainty principle as described in theorem \ref{theo:Heisenberg}. The wavelet transform is a rather new and superior transform in signal processing compared to the Fourier transform.
\\ \\
*** Her kan wavelet-transformationen fra kapitel 12 indsættes, hvis vi ikke vil have det i et separat kapitel. ***
\\ \\
As described in the discussion in chapter \ref{ch12} the final system could also be using different types of filters. Apart from the methods being used in the system, future implications also involve designing an app capable of performing the analysis, recognizing the exact frequencies in a rather noisy environment at certain times with respect to a given number of beats per minute, and finally showing the result in a staff system, which the musician may edit, save and print. Naturally, such system is highly desired by musicians all over the world, and big companies designing music notation softwares (such as the Sibelius software from Avid) are obviously working hard to design this kind of system. Therefore, the need for and range of future implications of the system described in this project are obviously huge.