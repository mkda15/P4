\section{The Discrete Fourier Transform} \label{sec:DFT}
The discrete Fourier transform (DFT) is the discrete-time variant of the continuous Fourier transform and can be used to transform samples from the time domain into the frequency domain. Multiple parts of the DFT are analogue to the continuous version. This section is inspired by the lecture notes from the course \textit{Anvendt Harmonisk Analyse} and \cite{page 252-264, FSP}.
\\ \\
Let a function in time $x(t)$ be a continuous signal. As defined in section \ref{sec:cont_Fourier_Transform}, the Fourier transform of $x(t)$ is:
\begin{align*}
\mathcal{F}\{x(t)\}(\omega) = X(\omega) = \int_{-\infty}^\infty x(t)\text{e}^{-j\omega t} dt.
\end{align*}

If the function is sampled $N$ times then the samples may be denoted $x[0],x[1],\dots,x[N-1]$. The discrete version is then:
\begin{align} \label{eq:DFT_vectors}
X[k] = \sum_{n=0}^{N-1} x[k]\text{e}^{-j\frac{2\pi}{N}kn} = \sum_{n=0}^{N-1} x[k] W_N^{kn}, \quad k = 0, 1, \dots, N-1,
\end{align}

where $W_N$ is the twiddle factor, which is defined as:
\begin{align*}
W_N = \text{e}^{-j\frac{2 \pi}{N}}, \quad W_N^{kn} = \text{e}^{-j\frac{2 \pi}{N}kn}.
\end{align*}

With the column vectors $\textbf{X} = [X[0] \ X[1] \ \dots \ X[N-1]]^T$ and $\textbf{x} = [x[0] \ x[1] \ \dots \ x[N-1]]^T$ then \eqref{eq:DFT_vectors} can be represented in matrix form:
\begin{align} \label{eq:DFT_matrix}
	\begin{bmatrix}
		X[0]\\ X[1]\\ X[2]\\ X[3] \\ \vdots \\ X[N-1]
	\end{bmatrix}
	=
	\begin{bmatrix}
		1 & 1 	& 1   	& \hdots & 1\\
		1 & W_N 	& W_N^2 	& \hdots & W_N^{N-1} \\
		1 & W_N^2	& W_N^4	& \hdots & W_N^{N-2} \\
		1 & W_N^3	& W_N^6	& \hdots & W_N^{N-3} \\
		\vdots & \vdots & \vdots & \ddots & \vdots \\
		1 & W_N^{N-1}	& W_N^{2(N-1)}	& \hdots & 						W_N^{(N-1)^2} \\
	\end{bmatrix}
	\begin{bmatrix}
		x[0]\\ x[1]\\ x[2]\\ x[3]\\ \vdots \\ x[N-1]
	\end{bmatrix}
	\Rightarrow
	\textbf{X} = F\textbf{x},
\end{align}

where $F$ is the $N \times N$ matrix containing the twiddle factors.
\\ \\
Since $W_N^N = \text{e}^{-j2\pi} = 1$ then $W_N$ is a unit root of order $N$. Let $F^*$ be the adjoint matrix of $F$, that is $(F^*)_{in} = \overline{F_{ni}}$. Then:
\begin{align*}
\sum_{i=0}^{N-1} F_{mi} (F^*)_{in} &= \sum_{i=0}^{N-1} W_N^{mi} (\overline{W_N})^{ni} = \sum_{i=0}^{N-1} \text{e}^{-j\frac{2\pi}{N}mi} \text{e}^{j\frac{2\pi}{N}ni} = \sum_{i=0}^{N-1} \text{e}^{-j\frac{2\pi}{N}(m-n)i} \\ &=
\begin{cases}
N, &m = n \\
\dfrac{\text{e}^{-j\frac{2\pi}{N}(m-n)N}-1}{1-\text{e}^{-j\frac{2\pi}{N}(m-n)}}, \quad &m \neq n
\end{cases} =
\begin{cases}
N, &m = n \\
0, \quad &m \neq n
\end{cases},
\end{align*}

which follows from the fact that the sum is a geometric sum:
\begin{align*}
\sum_{i=0}^{N-1} r^i = \dfrac{r^N-1}{1-r}.
\end{align*}

Therefore, $F^*F = N \cdot I$, where $I$ is the identity matrix. Hence
\begin{align*}
F^{-1} = \dfrac{1}{N} F^*.
\end{align*}

$F$ is thus a unitary operator (up to scaling) known as the DFT operator, which preserves the Euclidean norm (up to scaling) and $\frac{F}{\sqrt{N}}$ is a unitary matrix. Therefore, Plancherel's equality holds \cite{page 258, FSP}:
\begin{align*}
\|\textbf{x}\|^2 = \sum_{n=0}^{N-1} |x[n]|^2 = \dfrac{1}{N} \sum_{k=0}^{N-1} |X[k]|^2 = \dfrac{1}{N} \|\textbf{X}\|^2 = \dfrac{1}{N} \|F\textbf{x}\|^2.
\end{align*}

This is the analogue of Plancherel's equality stated in theorem \ref{theo:Plancherel}. This leads to the definition of the DFT \cite{page 253, FSP}.

\begin{definition}[The Discrete Fourier Transform] \label{def:DFT}
The discrete Fourier transform $X[k]$ of a length-$N$ sequence $x$ is an invertible and linear transform $\mathcal{F}: \mathbb{C}^N \to \mathbb{C}^N$ defined as
\begin{align} \label{eq:def_DFT}
\mathcal{F}\{\textbf{x}\}[k] = X[k] = \sum_{n=0}^{N-1} x[n] \text{e}^{-j\frac{2 \pi}{N}kn} = \sum_{n=0}^{N-1} x[n] W_N^{kn}, \quad k \in \mathbb{Z},
\end{align}

and such that $\textbf{X} = F\textbf{x}$ with the column vectors $\textbf{X}, \textbf{x} \in \mathbb{C}^N$ and $F$ as the $N\times N$ matrix in \eqref{eq:DFT_matrix}. This is called the spectrum of $x$. \\
The inverse DFT of a length-$N$ sequence $X$ is $\textbf{x} = F^{-1} \textbf{X} = \frac{1}{N} F^* \textbf{X}$ such that
\begin{align*}
x[n] = \dfrac{1}{N}\sum_{k=0}^{N-1} X[k] \text{e}^{j\frac{2 \pi}{N}kn} = \dfrac{1}{N}\sum_{k=0}^{N-1} X[k] W_N^{-kn}, \quad n \in \mathbb{Z}.
\end{align*}

The DFT and the inverse DFT can be denoted as a pair as
\begin{align*}
x[n] \xleftrightarrow{\text{DFT}} X[k].
\end{align*}
\end{definition}

The DFT has to take into consideration that it is only fed a finite number of input data. Since this is the case, the DFT treats the data as if it were periodic. A sequence $x[n]$ can be considered $N$-periodic since:
\begin{align*}
	X[n+N]
	&= \sum_{k=0}^{N-1} x[k] \text{e}^{-j \frac{2 \pi}{N}  k (n		+N)} \\
	&= \sum_{k=0}^{N-1} x[k] \text{e}^{-j \frac{2 \pi k n}{N}}		\text{e}^{\frac{2 \pi}{N} kN} \\
	&= \sum_{k=0}^{N-1} x[k] \text{e}^{-j \frac{2 \pi}{N} k n} 		= X[n].
\end{align*}

Plancherel's equation holds since for $\textbf{X} = F\textbf{x}$ and $\textbf{Y} = F\textbf{y}$:
\begin{align*}
\sum_{n=0}^{N-1} x[n] \overline{y[n]} = \langle \textbf{x},\textbf{y} \rangle = \dfrac{1}{N} \langle F^*F\textbf{x},\textbf{y} \rangle = \dfrac{1}{N} \langle \textbf{X},\textbf{Y} \rangle = \dfrac{1}{N} \sum_{k=0}^{N-1} X[k] \overline{Y[k]}
\end{align*}

The DFT is the Fourier transform designed for a finite-length sequence.
It treats all finite-length sequences as a period of an infinite-length periodic sequence.
For this case the \textit{circular convolution operator} is defined, which is closely related to the convolution sum defined in definition \ref{def:convolution}. \\
%The circular convolution operator is the appropriate description of LSI systems on circularly extended finite-length input, as in the case of DFT. 
%\chr{Ikke glad for denne sætning}\frede{Det forstår jeg godt, da jeg ikke forstår den \texttrademark}
%". As we haveseen in Section 3.3.3, the circular convolution operator (3.71) is the appropriate description of LSI systems operating on circularly extended finite-length inputs"
The circular convolution use the modulo function $\langle k\rangle_N$. It is possible to write $k = LN+r$ for $0 \leq r < N$ and $L \in \mathbb{Z}$ in a unique way, and the modulo function finds the remainder $r$ from division of $k$ by $N$:
\begin{align*}
\langle k \rangle_N = r.
\end{align*}

Examples are that $\langle 12 \rangle_{10} = 2$ since $12 = 1 \cdot 10 + 2$ and that $\langle -12 \rangle_{10} = 8$ since $-12 = -2 \cdot 10 + 8$.

\begin{definition}[Circular convolution]
 	The circular convolution between length-$N$ sequences $h$ 		and $x$ is defined as
	\begin{align*}
	(Hx)[n] = (h \circledast x)[n] = \sum_{k = 0}^{N-1} x[k] 		h[\langle n-k\rangle_N] =\sum_{k = 0}^{N-1} x[\langle n-k		\rangle_N] h[k]
	\end{align*}
	where $H$ is the circular convolution operator associated 		with $h$.
\end{definition}

For a general $k\in\mathbb{Z}$ with $k = LN + r, 0 \leq r \leq N$:
\begin{align*}
X[k] = \sum_{n=0}^{N-1} x[n] W_N^{kn} = \sum_{n=0}^{N-1} x[n] W_N^{(LN+r)n} = \sum_{n=0}^{N-1} x[n] W_N^{rn} = X[\langle k\rangle_N].
\end{align*}

\begin{lemma}
For $\textbf{x} = [x[0] \dots x[N-1]]^T$ and $0 \leq L \leq N$, then:
\begin{align*}
y[n]  = x[\langle n-L\rangle_N] \xleftrightarrow{\text{DFT}} Y[k] = X[k] \text{e}^{-j\frac{2\pi}{N} kL}.
\end{align*}
\end{lemma}

\begin{proof}
The DFT of $y[n] = x[\langle n-L\rangle_N]$ is written as two sums, which utilizes the properties of the modulo function:
\begin{align*}
Y[k] = \sum_{n=0}^{N-1} x[\langle n-L\rangle_N] \text{e}^{-j\frac{2\pi}{N} kn} &= \sum_{n=0}^{L-1} x[\langle n-L\rangle_N]\text{e}^{-j\frac{2\pi}{N} kn} + \sum_{n=L}^{N-1} x[n-L]\text{e}^{-j\frac{2\pi}{N} kn} \\
&= \sum_{n=0}^{L-1} x[N-L+n]\text{e}^{-j\frac{2\pi}{N} kn} + \sum_{n=L}^{N-1} x[n-L]\text{e}^{-j\frac{2\pi}{N} kn} \\
&= \sum_{m=N-L}^{N-1} x[m]\text{e}^{-j\frac{2\pi}{N} k(m+L)} + \sum_{n=L}^{N-1} x[n-L]\text{e}^{-j\frac{2\pi}{N} kn}
\end{align*}

with $m = N - L + n$ so that $n = m + L - N$ in the last equation. Since the last sum
\begin{align*}
\sum_{n=L}^{N-1} x[n-L]\text{e}^{-j\frac{2\pi}{N} kn} = \sum_{m=0}^{N-1-L} x[m] \text{e}^{-j\frac{2\pi}{N}k(m+L)}
\end{align*}

with $m = n - L$ so that $n = m + L$, the two sums above may be gathered into one sum:
\begin{align*}
Y[k] &= \sum_{m=N-L}^{N-1} x[m]\text{e}^{-j\frac{2\pi}{N} k(m+L)} + \sum_{m=0}^{N-1-L} x[m] \text{e}^{-j\frac{2\pi}{N}k(m+L)}  \\
&= \sum_{m=0}^{N-1} x[m] \text{e}^{-j\frac{2\pi}{N}k(m+L)} = \sum_{m=0}^{N-1} x[m] \text{e}^{-j\frac{2\pi}{N}kL}\text{e}^{-j\frac{2\pi}{N}km} = X[k] \text{e}^{-j\frac{2\pi}{N}kL}.
\end{align*}
\end{proof}

This result is used to prove the following main theorem.

\begin{theorem} \label{theo:DFT_circ}
For $\textbf{x} = [x_0 \dots x_{N-1}]^T$ and $\textbf{y} = [y_0 \dots y_{N-1}]^T$, then:
\begin{align*}
\textbf{z} = \textbf{x} \circledast \textbf{y} \xleftrightarrow{\text{DFT}} Z[k] = X[k] Y[k].
\end{align*}
\end{theorem}

\begin{proof}
The following sums are finite and may therefore be interchanged:
\begin{align*}
Z[k] &= \sum_{n=0}^{N-1} z[n] \text{e}^{-j\frac{2\pi}{N}kn} = \sum_{n=0}^{N-1} \left( \sum_{m=0}^{N-1} x[m] y[\langle n-m\rangle_N] \right) \text{e}^{-j\frac{2\pi}{N}kn} \\
&= \sum_{m=0}^{N-1} x[m] \left( \sum_{n=0}^{N-1} y[\langle n-m\rangle_N]\text{e}^{-j\frac{2\pi}{N}kn} \right) = \sum_{m=0}^{N-1} x[m] \left( \text{e}^{-j\frac{2\pi}{N}km}Y[k] \right) \\
&= \left( \sum_{m=0}^{N-1} x[m] \text{e}^{-j\frac{2\pi}{N}km} \right)Y[k] = X[k] Y[k].
\end{align*}
\end{proof}

Theorem \ref{theo:DFT_circ} shows that the circular convolution in the time domain corresponds to multiplication in the frequency domain. Therefore, this theorem is analogous to part (d) in theorem \ref{theorem:fund_sym_Fourier}.

%The following definition is the background for an important property of the DFT \cite{Eigenfunctions}.
%
%\begin{definition}[Eigenfunction]
%If $D$ is a linear operator on a function space then $f \not\equiv 0$ is an eigenfunction for $D$ and $\lambda$ is the associated eigenvalue, whenever $Df = \lambda f$.
%\end{definition}
%
%A property of LSI systems is that they as eigen sequences $v[n]$ all have complex exponential sequences with modulus being unit length on the form $v[n] = \text{e}^{j \omega n}$ \cite{FSP}.\\ The same follows from Linear Time Invariant (LTI) systems \cite{DTSP}.
%%Egenskaver ved eigenfunctioner:
%The eigenfunction property of the complex exponential function for an LSI system can thus be demonstrated as follows. Consider a complex sequence of radian frequency $v[n] = \text{e}^{j\omega n} \text{ for } -\infty < n < \infty$.
%\\\\
%The corresponding output of an LTI system with impulse response $h[n]$ is
%\begin{align*}
%y[n] 
%&= \sum_{k=-\infty}^{\infty} h[k]\text{e}^{j\omega(n-k)} \nonumber \\ 
%&= \text{e}^{j\omega n} \left(\sum_{k=-\infty}^{\infty} h[k]\text{e}^{-j\omega k} \right).
%\end{align*}
%
%This leads to
%\begin{align}\label{eq:eigenfunction}
%y[n] = H(\text{e}^{j \omega}) \text{e}^{j\omega n}
%\end{align}
%
%for
%\begin{align*}
%H(\text{e}^{j\omega}) = \sum_{k=-\infty}^{\infty} h[k]\text{e}^{-j\omega k}.
%\end{align*}
%
%This follows from the properties of eigenfunctions. Consequently, this leads to $\text{e}^{j\omega n}$ being an eigenfunction of the system with the associated eigenvalue $H(\text{e}^{j\omega})$.
%\\\\
%From \eqref{eq:eigenfunction} it can be noted that $H(\text{e}^{j\omega})$ will describe the change in the complex amplitude of a complex exponential input signal as a function of the frequency $\omega$ \cite{DTSP}. \chr{Might like this sentence rewritten., og dette bliver gentaget i filter sektionen (ved ikke hvad sammenhængen det er brugt til der)}
%%The eigenvalue $H(e^{j\omega})$ is called the frequency response of the system.
%\\
%Since the represented sequence is $N$-periodic, the eigensequence should also be periodic with period $N$, which leads to the following:
%\begin{align*}
%v[n+N] = \text{e}^{j\omega (n+N)} = v[n] \Rightarrow \text{e}^{j\omega N} = 1
%\end{align*}
%
%with $\omega = \dfrac{2 \pi k}{N}$ for $k \in \mathbb{Z}$. From this the previously mentioned twiddle factor can be seen \cite{FSP}:
%\begin{align*}
%v[n] = e^{j\frac{2\pi}{N} kn} = W_N^{-kn}.
%\end{align*}
%
%The twiddle factor can be shown to be an eigensequence for the circular convolution operator.
%\begin{align*}
%	(h\circledast v)[n] 
%	= \sum_{i=0}^{N-i} v[\langle n-i \rangle_N ]h[i] 
%	= \sum_{i=0}^{N-1} W_N^{-k\langle n-i\rangle_N}h[i]
%	&= \sum_{i=0}^{N-1} W_N^{-k(n-i)}h[i]\\
%	&= \sum_{i=0}^{N-1} h[i] W_n^{ki} \, W_N^{-kn}
%	= \lambda[n] v[n]
%\end{align*}\frede{Hvad er indeks $i$?}
%
%This shows that a circular convolution operator $H$ has an eigensequence that is the twiddle factor. \martin{Mangler for dette afsnit: delene om LSI-systemer og sidste del af afsnittet om egenfunktioner omformuleres og kobles bedre sammen med resten af afsnittet (eller eventuelt slettes, hvis det ikke er relevant). Derudover bør der angives sider ved kildehenvisningerne, hvor det er muligt. \textregistered}.