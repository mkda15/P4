\section{The Continuous Fourier Transform} \label{sec:cont_Fourier_Transform}
This section deals with the continuous Fourier transform. In this project the Fourier transform will be the background for transforming a signal $f(t)$ represented in the time domain to a signal $F(\omega)$ represented in the frequency domain, which will be useful when making the spectrograms mentioned in chapter \ref{ch1}. The section is inspired by \cite{FourierTrans} and \cite{FAA}.
\\ \\
The Fourier transform is a way of expanding functions on all of $\mathbb{R}$ in the same way that the Fourier series are used to expand functions on a finite interval.

\begin{definition}[The Fourier transform] \label{def:Fourier_trans}
For a function $f \in \mathcal{L}^1$ the Fourier transform of $f$ is defined as follows:
\begin{align*}
\mathcal{F}\{f(t)\}(\omega) = F(\omega) = \int_{-\infty}^\infty f(t) \text{e}^{-j \omega t} dt.
\end{align*}
\end{definition}

\begin{definition}[The inverse Fourier transform] \label{def:InverseFourier_trans}
For $F \in \mathcal{L}^1$, the inverse Fourier transform of $F(\omega)$ is:
\begin{align*}
\mathcal{F}^{-1}\{F(\omega)\}(t) = f(t) = \frac{1}{2\pi} \int_{-\infty}^\infty F(\omega) \text{e}^{j \omega t} d\omega.
\end{align*}
\end{definition}

The Fourier transform $F(\omega)$ is a function of frequency and provides information on the amplitude and the phase of the signal at various frequencies. $F(\omega)$ may be written in polar coordinates as $F = |F|e^{j\theta}$ with the modulus $|F|$ representing the amplitude of the signal at a certain frequency $\omega$ and the phase shift $\theta = \arctan \left(\frac{\operatorname{Im}(F)}{\operatorname{Re}(F)} \right)$ at the frequency $\omega$.
\\ \\
The following definition is inspired by \cite{page 206, FAA} and shows a useful tool in both theoretical and applicational aspects.
\begin{definition}[The convolution integral] \label{def:Convol}
Let $f,g$ be functions defined on $\mathbb{R}$. Their convolution is the function $f*g$ defined by:
\begin{align*}
(f*g)(t) = \int_{-\infty}^\infty f(t-\tau) g(\tau) d\tau
\end{align*}

provided that the integral exists.
\end{definition}

\noindent
The following statement shows the symmetry between $f(t)$ and $F(\omega)$ and is inspired by \cite{page 214, FAA}.

\begin{theorem} \label{theorem:fund_sym_Fourier}
Let $f \in \mathcal{L}^1$. Then:

\begin{enumerate}[label=(\alph*)]
\item For any $a \in \mathbb{R}$: \label{theorem:fund_sym_Fourier_a}
\begin{align*}
\mathcal{F}\{f(t-a)\}(\omega) = \text{e}^{-ja\omega} F(\omega), \\
\mathcal{F}\{\text{e}^{jat}f(t)\}(\omega) = F(\omega - a).
\end{align*}

\item If $\delta > 0$ and $f_\delta(t)=\delta^{-1}f(t/\delta)$, then
\begin{align*}
\mathcal{F}\{f_\delta(t)\}(\omega) = F(\delta\omega), \\
\mathcal{F}\{f(\delta t)\}(\omega) = F_\delta(\omega).
\end{align*}

\item If $f$ is continuous and piecewise smooth and $f' \in L^1$, then
\begin{align*}
\mathcal{F}\{f'(t)\}(\omega) = j\omega F(\omega).
\end{align*}

On the other hand, if $t\cdot f(t)$ is integrable, then
\begin{align*}
\mathcal{F}\{tf(t)\} = j F'(\omega).
\end{align*}

\item If also $g \in L^1$, then
\begin{align*}
\mathcal{F}\{f*g\}(\omega) = F(\omega) \cdot G(\omega).
\end{align*}
\end{enumerate}
\end{theorem}

\begin{proof}
\begin{enumerate}[label=(\alph*)]
\item
By definitions \ref{def:Fourier_trans} and \ref{def:InverseFourier_trans} it follows that:
\begin{align*}
\mathcal{F}\{f(t-a)\}(\omega) = \int_{-\infty}^\infty \text{e}^{-j\omega t} f(t - a) dt = \int_{-\infty}^\infty \text{e}^{-j\omega t - j\omega a} f(t) dt = \text{e}^{-ja \omega} F(\omega).
\end{align*}

For the second equation:
\begin{align*}
\mathcal{F}\{\text{e}^{jat} f(t)\}(\omega) = \int_{-\infty}^\infty f(t) \text{e}^{jat} \text{e}^{-j \omega t} dt = \int_{-\infty}^\infty f(t) \text{e}^{j(a- \omega)t} dt = F(\omega - a).
\end{align*}

\item For the first equation, let $\tau = t/\delta$. Hence, $t = \delta \tau$ and $dt = \delta d\tau$, which gives:
\begin{align*}
\mathcal{F}\{f_\delta(t)\}(\omega) &= \int_{-\infty}^\infty f_\delta(t) \text{e}^{-j \omega t} dt = \int_{-\infty}^\infty \delta^{-1}f(\tau) \text{e}^{-j \omega \delta\tau} \delta d\tau \\
&= \int_{-\infty}^\infty f(\tau) \text{e}^{-j \omega \delta\tau} d\tau = F(\delta\omega).
\end{align*}

For the second equation, let $\tau = \delta t$. Hence, $t = \delta^{-1}\tau$ and $dt = \delta^{-1} d\tau$:
\begin{align*}
\mathcal{F}\{f(\delta t)\}(\omega) &= \int_{-\infty}^\infty f(\delta t) \text{e}^{-j \omega t} dt = \int_{-\infty}^\infty f(\tau) \text{e}^{-j \omega \delta^{-1}\tau} \delta^{-1} d\tau = F_\delta(\omega).
\end{align*}

\item Since $f' \in L^1$, the integral in the limit
\begin{align*}
\lim_{t \to \infty} f(t) = f(0) + \lim_{t\to\infty} \int_0^t f'(x) dx = f(0) + \int_0^\infty f'(x) dx
\end{align*}

is finite, and the whole limit is 0 since $f \in \mathcal{L}^1$ satisfies $\int_{-\infty}^\infty |f(t)| dt < \infty$. Likewise, $\displaystyle{\lim_{t \to -\infty} f(t) = 0}$, too. Integration by parts yields:
\begin{align*}
\mathcal{F}\{f'(t)\}(\omega) &= \int_{-\infty}^\infty \text{e}^{-j \omega t} f'(t) dt \\
&= \left[ \text{e}^{-j\omega t} f(t) \right]_{-\infty}^\infty - \int_{-\infty}^\infty (-j \omega) \text{e}^{-j \omega t} f(t) dt \\
&= \int_{-\infty}^\infty (j \omega) \text{e}^{-j \omega t} f(t) dt = j\omega F(\omega)
\end{align*}

since $|\text{e}^{-j\omega t}| = 1 \ \forall \ t$ in the second equation. If $tf(t)$ is integrable then:
\begin{align*}
\mathcal{F}\{tf(t)\}(\omega) = \int_{-\infty}^\infty \text{e}^{-j \omega t} t f(t) dt = j \dfrac{d}{d\omega} \int_{-\infty}^\infty \text{e}^{-j \omega t} f(t) dt = j F'(\omega)
\end{align*}

Note that $t \text{e}^{-j \omega t} = j \dfrac{d}{d\omega} \text{e}^{-j\omega t}$.

\item Finally, by definition \ref{def:Convol} it follows that:
\begin{align*}
\mathcal{F}\{f*g\}(\omega) &= \int_{-\infty}^\infty \int_{-\infty}^\infty \text{e}^{-j \omega t} f(t - \tau) g(\tau) d\tau dt \\
&= \int_{-\infty}^\infty \int_{-\infty}^\infty \text{e}^{-j\omega(t-\tau)} f(t-\tau) \text{e}^{-j\omega \tau} g(\tau) dt d\tau \\
&= \int_{-\infty}^\infty \int_{-\infty}^\infty \text{e}^{-j\omega z} f(z) \text{e}^{-j\omega \tau} g(\tau) dz d\tau, \quad z = t - \tau \\
&= \int_{-\infty}^\infty \text{e}^{-j\omega z} f(z) dz \int_{-\infty}^\infty \text{e}^{-j\omega \tau} g(\tau) d\tau \\
&= F(\omega) \cdot G(\omega).
\end{align*}
\end{enumerate}
\end{proof}

\noindent
In all simplicity, part (a) of this theorem shows that translating a function (that is, a delay) corresponds to multiplying its Fourier transform by an exponential (which is equivalent to a phase shift since $F$ may be written in the polar coordinates $F = |F|\text{e}^{j\theta}$) and vice versa; part (b) shows that dilating a function by $\delta$ corresponds to dilating its Fourier transform by $1/\delta$ and vice versa; part (c) shows that differentiating a function corresponds to multiplying its Fourier transform by the coordinate variable and vice versa; and by definition \ref{def:Convol} it is shown that a convolution in the time domain is equivalent to multiplying in the frequency domain and vice versa \cite{page 215, FAA}.
\\ \\
A main result with regards to the Fourier transform is the following theorem, which is closely related to corollary \ref{coro:conv_Fourier_coeff} from the Fourier series \cite{page 217, FAA}.

\begin{theorem}[The Riemann-Lebesgue Lemma]
If $f \in \mathcal{L}^1$, then $F(\omega) \to 0$ as $\omega \to \pm \infty$.
\end{theorem}

\begin{proof}
$f$ is supposed to be a step function such that $f(t) = \sum_{j=1}^k c_j \phi_j(t)$. $\phi(t)$ is defined as:
\begin{align*}
\phi(t) =
	\begin{cases}
		1 \quad \text{for } |t| < a \\
		0 \quad \text{elsewhere}
	\end{cases}
\end{align*}

Hence, each $\phi_j$ equals 1 on a bounded interval $|t - t_j| < a_j$ and 0 elsewhere. The Fourier transform of $\phi(t)$ is:
\begin{align*}
\mathcal{F}\{\phi(t)\}(\omega) = \int_{-a}^a \text{e}^{-j\omega t} dt = \dfrac{e^{-ja\omega} - \text{e}^{ja\omega}}{-j\omega} = 2\dfrac{\sin(a\omega)}{\omega}.
\end{align*}

By theorem \ref{theorem:fund_sym_Fourier}\ref{theorem:fund_sym_Fourier_a}, the Fourier transform of $\phi_j(t)$ is:
\begin{align*}
\mathcal{F}\{\phi_j(t)\}(\omega) = 2 \omega^{-1} \text{e}^{-jt_j\omega} \sin(a_j\omega),
\end{align*}

which vanishes when $\omega \to \infty$, and hence so does $F(\omega)$.
\\
In the general case it is possible to find a sequence $\{f_n\}, f \in \mathcal{L}^1$, of step functions such that
\begin{align*}
\int_{-\infty}^\infty |f_n(t) - f(t)| dt \to 0.
\end{align*}

When $f$ is Riemann integrable this is simply a restatement of the fact that the integral of $f$ is the limit of Riemann sums. When $f$ is Lebesgue integrable the proof requires results from integration theory, which are omitted in this project. Since $|e^{j\omega t}| = 1$, the integral in the Fourier transform converges absolutely for all $\omega$ and defines a bounded function of $\omega$:
\begin{align*}
|F(\omega)| \leq \int_{-\infty}^\infty |f(t)| dt,
\end{align*}

which means that:
\begin{align*}
\sup_{\omega} |F_n(\omega) - F(\omega)| \leq \int_{-\infty}^\infty |f_n(t) - f(t)| dt \to 0 \quad \text{as } n \to \infty.
\end{align*}

Therefore, $F_n(\omega) \to F$ uniformly, and since each $F_n(\omega)$ vanishes at infinity, then so does $F(\omega)$.
\end{proof}

Moreover, the Fourier transform $F(\omega)$ is well-defined for $f \in \mathcal{L}^1$ since
\begin{align*}
|F(\omega)| \leq \int_{-\infty}^\infty |f(t)| \cdot |\text{e}^{j \omega t}| dt = \|f\|_{\mathcal{L}^1} < \infty.
\end{align*}

However, $F$ is not guaranteed to be in $\mathcal{L}^1$, which means the inverse Fourier transform might not be applicable. This is the reason why the space $\mathcal{L}^2$ also plays a significant role for Fourier transforms. If $f, g, F, G \in \mathcal{L}^1$ then $f$, $g$, $F$ and $G$ are also in $\mathcal{L}^2$ \cite{page 219, FAA}. By the definitions of the dot product and the inverse Fourier transform it follows that
\begin{align*}
2\pi \langle f,g \rangle = 2\pi \int_{-\infty}^\infty f(t) \overline{g(t)} dt &= \int_{-\infty}^\infty \int_{-\infty}^\infty f(t) \overline{\text{e}^{j\omega t} G(\omega)} d\omega dt \\
&= \int_{-\infty}^\infty \int_{-\infty}^\infty f(t) \text{e}^{-j\omega t} \overline{G(\omega)} dt d\omega \\
&= \int_{-\infty}^\infty F(\omega) \overline{G(\omega)} d\omega = \langle F,G \rangle.
\end{align*}

This is the analog of Parseval's equation in theorem \ref{theo:Fourier_series_Parseval} \cite{page 221, FAA}. Therefore, the Fourier transform preserves inner products up to a factor of $2\pi$. By taking $f = g$ it is seen that:
\begin{align*}
2\pi \langle f,f \rangle = \langle F,F \rangle \Rightarrow 2\pi \|f\|^2 = \|F\|^2.
\end{align*}

If $f$ is an arbitrary function in $\mathcal{L}^2$ it is possible to find a sequence $\{f_n\}$ such that $f_n$ and $F_n$ are in $\mathcal{L}^1$ and $f_n$ converges to $f$ in $\mathcal{L}^2$ \cite{page 82, FAA}. Since $\|F_n - F_m\|^2 = 2\pi\|f_n - f_m\|^2$, which converges to 0 as $m,n$ converges to $\infty$, $\{F_n\}$ is a Cauchy sequence in $\mathcal{L}^2$, which has a limit since $\mathcal{L}^2$ is complete. This limit only depends on $f$ and not on $\{f_n\}$, and it is simply defined to be the Fourier transform $F$. The domain of the Fourier transform is thus extended to include all of $\mathcal{L}^2$. This extension still preserves the norm and inner product up to a factor of $2\pi$, and it also satisfies the basic properties of the Fourier transform stated in theorem \ref{theorem:fund_sym_Fourier}. These results are stated below \cite{page 222, FAA}.

\begin{theorem}[The Plancherel Theorem] \label{theo:Plancherel}
The Fourier Transform, defined originally on $\mathcal{L}^1 \cap \mathcal{L}^2$, extends uniquely to the invertible map $\mathcal{F}: \mathcal{L}^2 \to \mathcal{L}^2$, where $F = \mathcal{F}\{f\}, G = \mathcal{F}\{g\}$, that satisfies:
\begin{align*}
\langle F, G \rangle = 2\pi \langle f,g \rangle, \\
\|F\|^2  = 2\pi \|f\|^2.
\end{align*}
for all $f,g \in \mathcal{L}^2$ \martin{Disse ligninger kaldes også Parsevals ligninger. Er det tydeligt nok, hvad der menes? \textregistered}.
\end{theorem}