\section{Recording of audio file}
(maybe for appendix)\\
For both unit tests and test on the final system a known audio file are needed as specified. This section accounts for the process of recording the audio in the acoustic lab.
\subsection{Description}
Four different audio files are needed as a result of the recording. The audio files are to exemplify music play by a guitar.    
\begin{itemize}
\item[1.] A single note (this has to be a heigh frequent tone, for the overtones to be farther from one another)
\item[2.] XX seconds melody without cords  
\item[3.] second file with additive noise   
\item[4.] XX seconds melody with cords 
\end{itemize} 
It is important to know the exact signal on each file, in order to use the files as reference points for the validation tests of the system. Due to this the recording of the audio will take place in the anechoic room provided by the acoustic laboratory at AAU.\\
Noise will be recorded separately in the anechoic room and afterwards added to the signal digitally. This is to reduce the amount of equipment and persons in the anechoic room which create unwanted reflections.  \\  
Each type of file will be recorded three$(?)$ times. This is done to verify the note played and minimize the human mistakes    
\\
\\
\textbf{Anechoic room}  \\
An anechoic room is designed to absorb all reflections from sound or electromagnetic waves. Further the room is isolated from exterior noise. Which makes it possible to only record or measure the exact sound that is coming from the instrument, without the presence of interfering reflections. \\ The specific room at AAU is build as a box inside a box. The inner box is placed on rubber suspensions and the inner walls are covered by sound absorbing wedges of about 0.4m length, due to requirements for anechoic performances. The inside dimensions of the room are 4.5m times 5.0m with a height of 4.0m.\cite{anechoic}

\subsection{Procedure}
The arrangement of the recording are sketched on figure \ref{}. Specific steps of the recording are listed below. \\

... insert drawing ... \\
\begin{itemize}
\item[1.] 
\end{itemize}
 



\subsection{Source of error}
\begin{itemize}
\item[-] Internal noise, from mic 
\item[-]	 Not playing exactly the not we want to play. 
\end{itemize}

\subsection{Evaluation}
 

