\chapter{Conclusion}
In this report, the theoretical background of discrete-time systems, the Fourier transform, sampling theory, and filtering has been described and applied to create a system capable of analysing a recorded audio signal in time and frequency simultaneously.
\\
The audio files used in this report contain music and noise recorded separately in an anechoic room and has been sampled according to the presented theory. The frequencies of the audio files has been analysed, and a bandpass filter with finite impulse response of type 1 has been designed for the purpose of filtering out the noise added to the recorded signal. The window method has been used to create the filter by using a Kaiser window with a window length of 2766 in order to meet the given specifications. As shown, the filter is capable of removing all frequencies outside the specified passband. The theory of the Fourier transform provides the background for the short-time Fourier transform, which has been used to create a spectrogram as the output of the simultaneous time and frequency analysis from the filtered signal. However, due to Heisenberg's uncertainty principle, the relation between the temporal and spectral resolutions is lower bounded, and arbitrarily good resolutions in both time and frequency are therefore not possible. From theoretical studies and experimental tests a Blackman window of length $2^{12}$ and $50\%$ overlap was found to provide a reasonable accuracy in both frequency and time.
\\ \\
It is documented that the system successfully detects the right frequency of a single tone within a range of $\pm 5$ Hz, which complicates the detection of low frequency tones. The most significant frequency is mostly a corresponding harmonic of the tone, which claims certain demands to an algorithm capable of recognising the actual tone. Furthermore, the system is limited to only being able to detect single tones, and it has therefore not been possible to detect chords. Therefore, the system is also capable of recognising single tones being played in a melody. Due to lacking implementation of a time detection algorithm it has not been possible to evaluate the temporal resolution of the spectrograms.
\\
By varying the signal-to-noise ratio within the frequency  passband it has been found that a very low SNR is acceptable for a wide spread noise. An $E_2$ was detected above an SNR of -41.89 dB and -62.46 dB for combined noise of clapping, singing, folding paper and background noise and white noise, respectively. However, a limitation arises when the noise is concentrated within a few frequencies and the amplitude of the noise exceeds the amplitude of the desired frequency. In this case an $E_2$ was detected above an SNR of 13.06 dB.
\\ \\
It is concluded that the final system transcribes single tone music to a note sheet in non-real-time with a frequency accuracy complicated for low frequency tones but without proper verification of temporal accuracy. Further implications involve deducing the right tone from the detected frequencies and printing these as notes in a staff system.

 