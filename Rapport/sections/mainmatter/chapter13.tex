\chapter{Conclusion}
Through this report is it documented how an application for generating a spectrogram corresponding to an audio file can be carried out.\\
The system takes an recorded input signal,  the signal is filtered to reduce noise, next the signal is transformed to the frequency domain to be represented by a spectrogram, from which the significant frequencies to a given time is identified and presented by a plot. (The intention of this output is the possibility to recognize the tones played - in order to compute a corresponding note sheet.)\\
\\
To reduce additive noise from the signal a  filter was designed as a bandpass type 1 FIR filter with use of a kaiser window which provides a window length of 2766 on behalf of given specification. It is shown that the filter succeeds in removing all frequencies outside the passband as intended.\\
The discrete Fourier transform is used as a mathematical tool present the frequency spectrum of the signal. In particular the short-time-Fourier transform was used to compute a spectrogram. From theoretical studies and experimental tests a Kaiser window of length $2^{12}$ with $\beta = 4$ and $50\%$ overlap was found to provide a reasonable accuracy in both frequency and time, considering Heisenberg uncertainty principle.\\
\\
It is documented that the system successfully detects the right frequency of a single tone within a range of $\pm 10$ Hz. Though it is mostly an corresponding harmonic of the tone that appears as the most significant frequency, which claims  certain demands to an eventual recognising algorithm. Further it is clarified that the application is limited to detect single tones and not chords. \\
By varying the signal-to-noise ratio within the frequency  passband it was found that for a wide spread noise a very low SNR is acceptable.Though a limitation arise when the noise is concentrated in few frequencies. \\ 
 \\
From this it is concluded that the application provides an acceptable output - with few limitations - under ideal circumstances according to the specifications for the system. (Though it takes further musical knowledge to deduce the right tone from the detected frequencies.)
\\ \\
*** Martins forslag: ***
In this report, the theoretical background of discrete-time systems, mathematics, the Fourier transform, sampling theory, and filtering has been described and applied to create a system capable of analysing a recorded audio file in time and frequency simultaneously.
\\
The audio files used in this report contain music and noise recorded separately in an anechoic room and has been sampled according to the presented theory. The frequencies of the audio files has been analysed, and a bandpass filter with finite impulse response of type I has been designed for the purpose of filtering out the noise added to the recorded signal. By the window method different windows has been used to create the filter, and the final filter was created by using a Kaiser window with a window length of 2766 in order to meet the given specifications. As shown, the filter is capable of removing all frequencies outside the specified passband. The theory of the Fourier transform provides the background for the short-time Fourier transform, which has been used to create a spectrogram as the output of the simultaneous time and frequency analysis from the filtered signal. However, due to Heisenberg's uncertainty principle, the relation between the resolutions of time and frequency is lower bounded, and arbitrarily good resolutions in both time and frequency are therefore not possible. From theoretical studies and experimental tests a Kaiser window of length $2^{12}$ with $\beta = 4$ and $50\%$ overlap was found to provide a reasonable accuracy in both frequency and time.
\\ \\
It is documented that the system successfully detects the right frequency of a single tone within a range of $\pm 10$ Hz. However, the most significant frequency is mostly a corresponding harmonic of the tone, which claims certain demands to an algorithm capable of recognizing the actual tone. Furthermore, the system has been limited to only being able to detect single tones, and it has therefore not been possible to detect chords. Therefore, the system is also capable of recognizing single tones being played in a melody.
\\
By varying the signal-to-noise ratio within the frequency  passband it has been found that a very low SNR is acceptable for a wide spread noise. However, a limitation arise when the noise is concentrated within a few frequencies and the amplitude of the noise exceeds the amplitude of the desired frequency.
\\ \\
It is concluded that the final system provides an acceptable output with only few limitations when it is applied under ideal circumstances as described in the specifications of the system. However, further implications involve deducing the right tone from the detected frequencies and printing these as notes in a staff system.

 