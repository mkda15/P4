\chapter{Conclusion}
Through this report is it documented how an application for generating a spectrogram corresponding to an audio file can be carried out.\\
The system takes an recorded input signal,  the signal is filtered to reduce noise, next the signal is transformed to the frequency domain to be represented by a spectrogram, from which the significant frequencies to a given time is identified and presented by a plot. (The intention of this output is the possibility to recognize the tones played - in order to compute a corresponding note sheet.)\\
\\
To reduce additive noise from the signal a  filter was designed as a bandpass type 1 FIR filter with use of a kaiser window which provides a window length of 2766 on behalf of given specification. It is shown that the filter succeeds in removing all frequencies outside the passband as intended.\\
The discrete Fourier transform is used as a mathematical tool present the frequency spectrum of the signal. In particular the short-time-Fourier transform was used to compute a spectrogram. From theoretical studies and experimental tests a Kaiser window of length $2^{12}$ with $\beta = 4$ and $50\%$ overlap was found to provide a reasonable accuracy in both frequency and time, considering Heisenberg uncertainty principle.\\
\\
It is documented that the system successfully detects the right frequency of a single tone within a range of $\pm 10$ Hz. Though it is mostly an corresponding harmonic of the tone that appears as the most significant frequency, which claims  certain demands to an eventual recognising algorithm. Further it is clarified that the application is limited to detect single tones and not chords. \\
By varying the signal-to-noise ratio within the frequency  passband it was found that for a wide spread noise a very low SNR is acceptable.Though a limitation arise when the noise is concentrated in few frequencies. \\ 
 \\
From this it is concluded that the application provides an acceptable output - with few limitations - under ideal circumstances according to the specifications for the system. (Though it takes further musical knowledge to deduce the right tone from the detected frequencies.)     \\

 