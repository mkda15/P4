\section{Filter design}
In this section the process of designing digital filter are described. There will be a distinguished between techniques for IIR and FIR filters..\\
As described ideal filters are not computable hence the process of designing a filter is based on approximation of the frequency response made by computable polynomials.  


\subsection{IIR-filter}
One common way of designing an IIR-filter is to design the discrete-time filter from a corresponding continuous-time system. The idea follows four main steps. 
\begin{itemize}
\item[1.] Specify the properties desired for the filter.
\item[2.] Hereby compute a "prototype" by a continuous-time system $H_c(j\Omega)$ that approximates the given properties.
\item[3.] Transform the "prototype" into the digital filter $H(?)$ by...
\item[4.] Implementation of the filter. 
\end{itemize}
The properties of a system are specified on behave of the desired application, considering what frequencies are to pass the filter ideally. Further it is important to specify how much the filter are allowed to vary from the ideal properties, from which it will vary. \\
Properties for an approximation to a lowpass filter could be defined by bounding the magnitude within $\pm \ \delta_1$ of unity in a limited frequency band $0 \leq \Omega \leq \Omega_p $ and less than $\delta_2$ in the frequency band $\Omega_s \leq \Omega$. The magnitude $|H_c(j\Omega)|$ pleasing the properties is illustrated on figure \ref{??}\\
The desired gain of the magnitude is often expressed in dB, where $20\log(1)=0[\text{dB}]$ \trine{is gain of magnitude a thing?} \\
The transition off nonzero width between the cutoff frequency of passband and stopband is necessary in order for the system to be realizable. \\
\\
Indsæt figur ala fig 7.2a side 441 DTSP\\
\\
