\section{Signal-to-noise ratio}
In this section the term signal-to-noise ratio (SNR), is described.
It is further used to analyze the additive noise, on the recorded audio.
Signal-to-noise ratio (SNR) is a measure of the amount of noise in a signal. 
The SNR is defined as
\begin{equation}\label{eq:SNR}
SNR=\frac{\sigma_{signal}^2}{\sigma_{noise}^2}
\end{equation}
where $\sigma_{signal}^2$ is the variance of the wanted signal and $\sigma_{noise}^2$ is the variance of the noise.

To try and determine how small the SNR is allowed to be, the recorded music will be added noise in various level such that the SNR is changed. 
The peak detection algorithm is then used to find peaks, and it is observed when it does not manage to find the right peak as a consequence of the excessive noise in the signal.