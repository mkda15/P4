\chapter{Problem analysis} \label{ch1}
In this chapter an introduction will lead to a discussion of problems encountered by a musician when music is to be transcribed. The importance of these problems is analysed and the need for a solution assessed whereafter existing and possible solutions with roots in mathematics will be presented. The chapter will conclude with a \textit{problem statement} which will form the basis for the rest of the project.
\section{Introduction}
Some musicians are capable of playing without reading music written on a sheet of music, and they invent new music through a creative process where they play by ear and do not have the need to write down their compositions. Creative thinking and processes are however interruptable and this can be problematic when the need for transcribing a composition - so as to remember or convey it to others - to a sheet of music arises. This may ruin the creative proces by exactly interrupting it because one needs to concentrate on the transcription.\\\\
To ease the creative process of a musician it is imaginable that some kind of automatic real time transcription of music to sheet music can be developed which will eliminate the problem of having to interrupt the aforementioned process.
\section{Problem analysis}
Although the first system of musical notation is the Sumerian system created 3,500 years ago music has existed for much longer. \cite{origins} This means that people have been playing music without any form of written music until then. Even with the creation of standardised musical notation musicianship on high level is possible without any form of education and/or skills in reading and writing music. As such there continue to exist musicians without aforementioned skills.\\\\
Inability to read or write music poses no problem in itself, as it doesn't necessarily hinder musical creativity. Problems however arise when the need to convey or remember musical compositions presents itself. Just as standardised languages make everyday communication and tasks easier a standardised musical notation is needed to convey compositions to others without anyone having to remember the compositions in their entirety. Standard musical notation has been created but has to be learned and research suggests that the ability to read and write music varies considerably from person to person and might even be affected by dyslexia. \cite{dyslexia} In short people are genetically diverse when it comes to learning to read music.\\\\
If musical transcription in the middle of a creative process potentially interrupts said process the effect would be increasingly strengthened by the inability to quickly perform transcriptions which in the worst case scenario would require another person to do the transcription. These consequences of the inability to read/write in musical notation may hinder the distribution of otherwise ingenious musical creations.
\subsection{Existing and possible solutions}
There has been undertaken plenty of research regarding automatic transcription of music to a note sheet and the material is to find in both books \cite{sol1} and articles \cite{sol2}. There also exists downloadable software which helps transcription by hand \cite{transcribe!}. The extensive research done stems from the applications extending to other areas than just transcribing music for musicians - making computers participate with human performances is another imaginable application. Although the research is extensive "[$\hdots$]the performance of transcription systems is still significantly below that of a human expert[$\hdots$]" \cite{future} and so there is room for improvement.\\\\
This project will focus on creating a solution with the use of mathematical tools to extract signal information from a digitalized analog signal of music so as to transcribe it to a note sheet. The viability of a solution in form of an algorithm based on the mathematical tools will be tested in a lab to illustrate the possibilities and limitations of said algorithm. \frede{Her skal  måske uddybes, når vi ved mere om vores løsning og metode.}
\subsection{Problem statement}
The above analysis is summarized.\\
When musicians compose new music they firstly play but it stunts their creativity when they need to stop to transcribe the music. There is furthermore not necessarily any correlation between musical talent and the ability to transcribe music. This makes it advantageous for many musicians to use a program which automatically transcribes music to a sheet of music. This leads to the following problem statement:
\begin{itemize}
\item[] \textit{How can an algorithm through the use of mathematical tools be designed to automatically transcribe music to a note sheet?}
\end{itemize}
\subsubsection{Limitations}
Given the time scope and the capability of collaborators of this project it is not expected to solve the above problem statement to full extent. Considering this a number of delimitations for the project will be decided upon in chapter \ref{ch3}.
%The music used for testing the algorithm will be simple melodies comtaining one note at a time (no chords) from one instrument at a time. This will allow insight into the methods used without complicating the recognition of notes by blending together different instruments and different sound frequencies.\\\\
%The music transcription will be done in non-real time - the designed algorithm will be used on a prerecorded piece of music. This makes it possible to run the algorithm without having to play the piece of music every time and furthermore makes it possible to neglect computational power and/or efficiency.\\\\
%As such the end product of this project will ideally be limited to being an algorithm based on mathematical tools which is capable of transcribing a prerecorded piece of simple music to a spectrogram. 
%\input{sections/mainmatter/chapter_1/_____.tex}
