\chapter{Problem analysis} \label{ch1}
In this chapter an introduction will lead to a discussion of problems encountered by a musician when music is to be transcribed. The importance of these problems is analysed and the need for a solution assessed whereafter existing and possible solutions with roots in mathematics will be presented. The chapter will conclude with a \textit{problem statement} which will form the basis for the rest of the project.
\section{Introduction}
Some musicians are capable of playing without reading music written on a sheet of music, and they invent new music through a creative process where they play by ear and do not have the need to write down their compositions. Creative thinking and processes are however interruptable and this can be problematic when the need for transcribing a composition - so as to remember or convey it to others - to a sheet of music arises. This may ruin the creative proces by exactly interrupting it because one needs to concentrate on the transcription.\\\\
To ease the creative process of a musician it is imaginable that some kind of automatic real time transcription of musical compositions can be developed which will eliminate the problem of having to interrupt the aforementioned process.
\section{Problem analysis}
Although the first system of musical notation is the Sumerian system created 3,500 years ago music has existed for much longer. \cite{origins} This means that people have been playing music without any form of written music until then. Even with the creation of standardised musical notation musicianship on high level is possible without any form of education and/or skills in reading and writing music. As such there continue to exist musicians without aforementioned skills.\\\\
Inability to read or write music poses no problem in itself, as it doesn't necessarily hinder musical creativity. Problems however arise when the need to convey or remember musical compositions presents itself. Just as standardised languages make everyday communication and tasks easier a standardised musical notation is needed to convey compositions to others without anyone having to remember the compositions in their entirety. Standard musical notation has been created but has to be learned and research suggests that the ability to read and write music varies considerably from person to person and might even be affected by dyslexia. \cite{dyslexia} In short people are genetically diverse when it comes to learning to read music.\\\\
If musical transcription in the middle of a creative process potentially interrupts said process the effect would be increasingly strengthened by the inability to quickly perform transcriptions which in the worst case scenario would require another person to do the transcription. These consequences of the inability to read/write in musical notation may hinder the distribution of otherwise ingenious musical creations.
\subsection{Existing and possible solutions}
Existing

%\input{sections/mainmatter/chapter_1/_____.tex}
