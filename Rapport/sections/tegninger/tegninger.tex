\title{Opgave 164 \\ \small e16g3113}
\date{\today}
\documentclass[12pt]{article}

\usepackage{amsmath}
\usepackage{amssymb}
\usepackage{amsthm}



\usepackage[
%  disable, %turn off todonotes
 colorinlistoftodos, %enable a coloured square in the list of todos
 textwidth=\marginparwidth, %set the width of the todonotes
 textsize=scriptsize, %size of the text in the todonotes
 ]{todonotes}
\usepackage{tikz}
\usetikzlibrary{shapes,arrows}

\usepackage[utf8]{inputenc}
\usepackage[danish,english]{babel}
\renewcommand\qedsymbol{$\blacksquare$}
\usepackage{lmodern}
\usepackage{textcomp}


\begin{document}

%setup
\tikzset{%
  block/.style    = {draw, thick, rectangle, minimum height = 3em,
    minimum width = 3em},
  sum/.style      = {draw, circle, node distance = 2cm}, % Adder
  input/.style    = {coordinate}, % Input
  output/.style   = {coordinate} % Output
}

%definerer labels 
\newcommand{\suma}{\Large$+$}
\newcommand{\inte}{$\displaystyle \int$}
\newcommand{\derv}{\huge$\frac{d}{dt}$}

\begin{figure}[h]
\centering
\begin{tikzpicture}[auto, thick, node distance=5cm, >=triangle 45]
\draw
% Drawing the blocks of first filter :
	node [input, name=input1] {} 		
	node [block, right of=input1] (sh) {S/H}
	node [block, right of=sh] (quant) {Quantification}
    node [output, name=output1, right of=quant]{}
    node at (input1) [above=5mm] {Mic};
;
% Joining blocks. 
% Commands \draw with options like [->] must be written individually

\draw[->](input1) -- node {Analogue input}(sh);
\draw[->](sh) -- node {discrete signal}(quant);
\draw[->](quant) -- node {digital input}(output1);
\filldraw[color=black,fill=white,thick](input1) circle (0.3);
\draw(-0.3,0.3) -- (-0.3,-0.3);
\draw (-1.5,0) sin (-1.40,0.5) cos (-1.30,0) sin (-1.20,-0.5) cos (-1.10,0) sin (-1,0.5) cos (-0.9,0) sin (-0.8,-0.5) cos (-0.7,0);

% Boxing and labelling noise shapers
\draw [color=gray,thick,dashed](-0.5,-1.5) rectangle (12,1.5);
\node at (-0.5,1.7) [above=5mm, right=0mm] {\textsc{ADC}};
\end{tikzpicture}
\caption{Basic block-diagram of the system}
\label{fig:model1}
\end{figure}



%%definerer labels 
%\newcommand{\suma}{\Large$+$}
%\newcommand{\inte}{$\displaystyle \int$}
%\newcommand{\derv}{\huge$\frac{d}{dt}$}
%
%\begin{figure}
%\centering
%\begin{tikzpicture}[auto, thick, node distance=2cm, >=triangle 45]
%\draw
%% Drawing the blocks of first filter :
%	node at (0,0)[right=-5mm] {}
%	node [input, name=input1] {} 
%	node [sum, right of=input1] (suma1) {\suma}
%	node [block, right of=suma1] (inte1) {\inte}
%         node at (6.8,0)[block] (Q1) {\Large $Q_1$}
%         node [block, below of=inte1] (ret1) {\Large $T_1$};
%% Joining blocks. 
%% Commands \draw with options like [->] must be written individually
%\draw[->](input1) -- node {$X(Z)$}(suma1);
%\draw[->](suma1) -- node {} (inte1);
%\draw[->](inte1) -- node {} (Q1);
%\draw[->](ret1) -| node[near end]{} (suma1);
%% Adder
%\draw
%	node at (5.4,-4) [sum, name=suma2] {\suma}
%    	% Second stage of filter 
%	node at  (1,-6) [sum, name=suma3] {\suma}
%	node [block, right of=suma3] (inte2) {\inte}
%	node [sum, right of=inte2] (suma4) {\suma}
%	node [block, right of=suma4] (inte3) {\inte}
%	node [block, right of=inte3] (Q2) {\Large $Q_2$}
%	node at (9,-8) [block, name=ret2] {\Large $T_2$}
%;
%% Joining the blocks of second filter
%\draw[->] (suma3) -- node {} (inte2);
%\draw[->] (inte2) -- node {} (suma4);
%\draw[->] (suma4) -- node {} (inte3);
%\draw[->] (inte3) -- node {} (Q2);
%\draw[->] (ret2) -| (suma3);
%\draw[->] (ret2) -| (suma4);
%
%% Third stage of filter:
%	% Defining nodes:
%\draw
%	node at (11.5, 0) [sum, name=suma5]{\suma}
%	node [output, right of=suma5]{}
%	node [block, below of=suma5] (deriv1){\derv}
%	node [output, right of=suma5] (sal2){}
%;
%% Joining the blocks:
%\draw[->] (suma2) -| node {}(suma3);
%\draw[->] (Q1) -- (8,0) |- node {}(ret1);
%\draw[->] (8,0) |- (suma2);
%\draw[->] (5.4,0) -- (suma2);
%\draw[->] (Q1) -- node {}(suma5);
%\draw[->] (deriv1) -- node {}(suma5);
%\draw[->] (Q2) -| node {}(deriv1);
%\draw[<->] (ret2) -| node {}(deriv1);
%\draw[->] (suma5) -- node {$Y(Z)$}(sal2);
%% Drawing nodes with \textbullet
%\draw
%	node at (8,0) {\textbullet} 
%	node at (8,-2){\textbullet}
%	node at (5.4,0){\textbullet}
%    	node at (5,-8){\textbullet}
%    	node at (11.5,-6){\textbullet}
%    	;
%% Boxing and labelling noise shapers
%\draw [color=gray,thick](-0.5,-3) rectangle (9,1);
%\node at (-0.5,1.5) [above=5mm, right=0mm] {\textsc{first-order noise shaper}};
%\draw [color=gray,thick](-0.5,-9) rectangle (12.5,-5);
%\node at (-0.5,-9) [below=5mm, right=0mm] {\textsc{second-order noise shaper}};
%\end{tikzpicture}
%
%\label{fig:free_body}
%\end{figure}
%
\end{document}