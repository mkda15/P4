\section{Vurdering}
På bagrund af graferne i sektion \ref{ch4_design} kan det konkluderes, at det designede FIR-filter formår at filtrere frekvensen $\omega_2=\frac{\pi}{2}$ fra det samplede signal $s[n]$ som ønsket ved anvendelse af orden større end $30$. \\
Figur \ref{fig:filter_rekt} illustrerer, at de opstillede specifikationer til en vis grad overholdes ved en orden på hhv. 30 og 100 og anvendelse af det rektangulære vindue. Det ses tydeligt, at transitionsbåndet bliver smallere jo højere orden der anvendes. De ripples, der forekommer i pasbåndet formindskes også, når ordenen øges. Dog forsvinder de ikke, da det er det rektangulære vindue, der anvendes. Som nævnt kan dette optimeres ved at ændre på vinduet, hvilket undersøges nærmere i næste sektion. \\
Figur \ref{filt signal i frekvens} illustrerer yderligere hvorledes den ønskede frekvens er fjernet uden at fjerne de omkringliggende frekvenser. Dog ses som resultat, at de forekommende rippels i pasbåndet, at amplituderesponsen på de tilbageværende frekvenser forvrænges en smule.    
Ved sammenligning af amplituderesponsen for det designede filter på figur \ref{fig:filter_rekt} og den ideelle amplituderespons skitseret på figur \ref{fig:ideel_amp_respons} vurderes det altså, at det designede filter kan optimeres yderligere ved anvendelse af et andet vindue. \\