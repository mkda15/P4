På bagrung af graferne i sektion \ref{resultater} kan det konkluderes at det designede FIR filter af orden ?? formår at filtrere frekvensen $\omega_2=\frac{\pi}{2}$ fra det samplede signal $s[n]$, som ønsket.\\
Figur \ref{amplitude respons} illustrerer at de opstillede specifikationer til en vis grad overholdes ved en orden på hhv. 30 og 100 og anvendelse af det reksangulare vindue. Dog ses det tydeligt at transitionsbåndet bliver smallere jo højere orden der anvendes. De ripples der forekommer i passbåndet formindske ligeså, som ordnen øges, dog forsvinder de ikke som resultat at at det er det rektangulere vindue der anvendes. Som nævnt kan dette optimeres ved at ændre vinduet, hvilket undersøges nærmere i næste afsnit. \\
Figur \ref{filt signal i frekvens} illustrere yderligere hvorledes den ønskede frekvens er fjernet, uden at fjerne de omkring liggende frekvenser. Dog ses som resultat af de forekommende  rippels i passbåndet at amplituderesponsen på de tilbageværende frekvenser forvrænges en smule.    
Ved sammenligning af amplituderesponsen for det designede filter \ref{?} og den ideelle amplituderespons, skitseret på figur \ref{?}, vurderes det altså at det designede filter kan optimeres yderligere. \\
          